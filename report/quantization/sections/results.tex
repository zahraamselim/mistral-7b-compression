\section{Results}

This section presents empirical results across three evaluation dimensions: computational efficiency (Section \ref{sec:efficiency}), task performance quality (Section \ref{sec:quality}), and RAG-specific capabilities (Section \ref{sec:rag}). All measurements are conducted on NVIDIA Tesla T4 (16GB VRAM) using Mistral-7B-v0.1 as the base model with standard quantization configurations (no calibration-aware optimization).

\subsection{Computational Efficiency Analysis}
\label{sec:efficiency}

\subsubsection{Latency Characterization}

Table \ref{tab:latency_detailed} presents comprehensive latency measurements across all quantization methods, decomposed into prefill (prompt processing) and decode (autoregressive generation) phases.

\begin{table}[H]
\centering
\caption{Latency measurements across quantization methods}
\label{tab:latency_detailed}
\small
\begin{tabular}{lccccc}
\toprule
\textbf{Method} & \textbf{TTFT} & \textbf{Decode} & \textbf{Prefill} & \textbf{Total} & \textbf{Speedup} \\
 & \textbf{(ms)} & \textbf{(ms/tok)} & \textbf{(ms)} & \textbf{(ms)} & \textbf{vs FP16} \\
\midrule
FP16 & & & & & 1.00$\times$ \\
NF4 & & & & & \\
GPTQ & & & & & \\
AWQ & & & & & \\
QuaRot & & & & & \\
HQQ & & & & & \\
KVQuant & & & & & \\
\bottomrule
\end{tabular}
\end{table}

\subsubsection{Throughput and Memory Efficiency}

Table \ref{tab:throughput_memory} characterizes sustained generation rates and memory consumption patterns.

\begin{table}[H]
\centering
\caption{Throughput and memory utilization}
\label{tab:throughput_memory}
\small
\begin{tabular}{lcccccc}
\toprule
\textbf{Method} & \textbf{Throughput} & \textbf{Model} & \textbf{Peak} & \textbf{Bits/} & \textbf{Memory} & \textbf{Compression} \\
 & \textbf{(tok/s)} & \textbf{Size (GB)} & \textbf{VRAM (MB)} & \textbf{Param} & \textbf{Efficiency} & \textbf{Ratio} \\
\midrule
FP16 & & 13.49 & & 16.0 & & 1.00$\times$ \\
NF4 & & & & 4.0 & & \\
GPTQ & & & & 4.0 & & \\
AWQ & & & & 4.0 & & \\
QuaRot & & & & 4.0 & & \\
HQQ & & & & 4.0 & & \\
KVQuant & & & & 16.0* & & \\
\bottomrule
\multicolumn{7}{l}{\footnotesize *KVQuant quantizes KV cache, not weights}
\end{tabular}
\end{table}

\subsubsection{Computational Efficiency Metrics}

Table \ref{tab:compute_efficiency} presents Model FLOPs Utilization and energy consumption estimates, revealing how effectively each method exploits available hardware capacity.

\begin{table}[H]
\centering
\caption{Computational efficiency and energy consumption}
\label{tab:compute_efficiency}
\small
\begin{tabular}{lcccc}
\toprule
\textbf{Method} & \textbf{FLOPs/tok} & \textbf{MFU} & \textbf{Energy} & \textbf{Energy} \\
 & \textbf{(GFLOPs)} & \textbf{(\%)} & \textbf{(mJ/tok)} & \textbf{Efficiency} \\
\midrule
FP16 & & & & 1.00$\times$ \\
NF4 & & & & \\
GPTQ & & & & \\
AWQ & & & & \\
QuaRot & & & & \\
HQQ & & & & \\
KVQuant & & & & \\
\bottomrule
\end{tabular}
\end{table}

\subsubsection{KV Cache Memory Analysis}

For methods with explicit KV cache management, Table \ref{tab:kv_cache_analysis} estimates cache sizes at different sequence lengths, demonstrating memory scaling behavior.

\begin{table}[H]
\centering
\caption{KV cache memory requirements by sequence length}
\label{tab:kv_cache_analysis}
\small
\begin{tabular}{lcccc}
\toprule
\textbf{Method} & \textbf{1K tokens} & \textbf{2K tokens} & \textbf{4K tokens} & \textbf{8K tokens} \\
 & \textbf{(MB)} & \textbf{(MB)} & \textbf{(MB)} & \textbf{(MB)} \\
\midrule
FP16 & & & & \\
NF4 & & & & \\
GPTQ & & & & \\
AWQ & & & & \\
QuaRot & & & & \\
HQQ & & & & \\
KVQuant & & & & \\
\bottomrule
\end{tabular}
\end{table}

\subsubsection{Batch Throughput Scaling}

Table \ref{tab:batch_throughput} examines throughput scaling across batch sizes, critical for understanding deployment capacity in production scenarios.

\begin{table}[H]
\centering
\caption{Batch throughput scaling (tokens/second)}
\label{tab:batch_throughput}
\small
\begin{tabular}{lcccc}
\toprule
\textbf{Method} & \textbf{Batch=1} & \textbf{Batch=2} & \textbf{Batch=4} & \textbf{Batch=8} \\
\midrule
FP16 & & & & \\
NF4 & & & & \\
GPTQ & & & & \\
AWQ & & & & \\
QuaRot & & & & \\
HQQ & & & & \\
KVQuant & & & & \\
\bottomrule
\end{tabular}
\end{table}

\subsubsection{Efficiency Summary and Pareto Analysis}

Table \ref{tab:efficiency_summary} consolidates key efficiency metrics for direct comparison.

\begin{table*}[H]
\centering
\caption{Consolidated efficiency metrics summary}
\label{tab:efficiency_summary}
\small
\begin{tabular}{lcccccccc}
\toprule
\textbf{Method} & \textbf{Latency} & \textbf{Throughput} & \textbf{TTFT} & \textbf{Memory} & \textbf{Model} & \textbf{MFU} & \textbf{Energy} & \textbf{Overall} \\
 & \textbf{(ms/tok)} & \textbf{(tok/s)} & \textbf{(ms)} & \textbf{(MB)} & \textbf{(GB)} & \textbf{(\%)} & \textbf{(mJ/tok)} & \textbf{Rank} \\
\midrule
FP16 & & & & & 13.49 & & & \\
NF4 & & & & & & & & \\
GPTQ & & & & & & & & \\
AWQ & & & & & & & & \\
QuaRot & & & & & & & & \\
HQQ & & & & & & & & \\
KVQuant & & & & & & & & \\
\bottomrule
\end{tabular}
\end{table*}

\subsection{Task Performance Analysis}
\label{sec:quality}

\subsubsection{Language Modeling: Perplexity}

Table \ref{tab:perplexity} presents perplexity measurements on WikiText-2, a foundational metric for assessing language modeling quality under compression.

\begin{table}[H]
\centering
\caption{Perplexity on WikiText-2 test set (100 samples)}
\label{tab:perplexity}
\small
\begin{tabular}{lcccc}
\toprule
\textbf{Method} & \textbf{Perplexity} & \textbf{Loss} & \textbf{Increase} & \textbf{Degradation} \\
 & & & \textbf{(abs)} & \textbf{(\%)} \\
\midrule
FP16 & 12.79 & & 0.00 & 0.00\% \\
NF4 & & & & \\
GPTQ & & & & \\
AWQ & & & & \\
QuaRot & & & & \\
HQQ & & & & \\
KVQuant & & & & \\
\bottomrule
\end{tabular}
\end{table}

\subsubsection{Commonsense Reasoning}

Table \ref{tab:commonsense} evaluates performance on commonsense reasoning benchmarks requiring world knowledge and intuitive understanding.

\begin{table}[H]
\centering
\caption{Commonsense reasoning performance (0-shot)}
\label{tab:commonsense}
\small
\begin{tabular}{lccccc}
\toprule
\textbf{Method} & \textbf{HellaSwag} & \textbf{ARC-Easy} & \textbf{ARC-Challenge} & \textbf{Average} & \textbf{Degradation} \\
 & \textbf{(acc\_norm)} & \textbf{(acc\_norm)} & \textbf{(acc\_norm)} & & \textbf{vs FP16} \\
\midrule
FP16 & 0.720 & 0.760 & 0.580 & 0.687 & 0.00\% \\
NF4 & & & & & \\
GPTQ & & & & & \\
AWQ & & & & & \\
QuaRot & & & & & \\
HQQ & & & & & \\
KVQuant & & & & & \\
\bottomrule
\end{tabular}
\end{table}

\subsubsection{Mathematical Reasoning}

Table \ref{tab:math_reasoning} assesses preservation of multi-step mathematical reasoning capabilities, often the most sensitive to quantization.

\begin{table}[H]
\centering
\caption{Mathematical reasoning performance on GSM8K (8-shot)}
\label{tab:math_reasoning}
\small
\begin{tabular}{lccc}
\toprule
\textbf{Method} & \textbf{Accuracy} & \textbf{Absolute} & \textbf{Relative} \\
 & \textbf{(exact match)} & \textbf{Drop} & \textbf{Degradation} \\
\midrule
FP16 & 0.360 & 0.000 & 0.0\% \\
NF4 & & & \\
GPTQ & & & \\
AWQ & & & \\
QuaRot & & & \\
HQQ & & & \\
KVQuant & & & \\
\bottomrule
\end{tabular}
\end{table}

\subsubsection{World Knowledge}

Table \ref{tab:world_knowledge} evaluates factual knowledge retention across 57 academic domains via MMLU.

\begin{table}[H]
\centering
\caption{World knowledge evaluation on MMLU (5-shot)}
\label{tab:world_knowledge}
\small
\begin{tabular}{lccc}
\toprule
\textbf{Method} & \textbf{Accuracy} & \textbf{Absolute} & \textbf{Relative} \\
 & & \textbf{Drop} & \textbf{Degradation} \\
\midrule
FP16 & 1.000 & 0.000 & 0.0\% \\
NF4 & & & \\
GPTQ & & & \\
AWQ & & & \\
QuaRot & & & \\
HQQ & & & \\
KVQuant & & & \\
\bottomrule
\end{tabular}
\end{table}

\subsubsection{Code Generation}

Table \ref{tab:code_generation} measures code synthesis quality via functional correctness on HumanEval.

\begin{table}[H]
\centering
\caption{Code generation performance on HumanEval (0-shot)}
\label{tab:code_generation}
\small
\begin{tabular}{lccc}
\toprule
\textbf{Method} & \textbf{Pass@1} & \textbf{Absolute} & \textbf{Relative} \\
 & & \textbf{Change} & \textbf{Change} \\
\midrule
FP16 & 0.050 & 0.000 & 0.0\% \\
NF4 & & & \\
GPTQ & & & \\
AWQ & & & \\
QuaRot & & & \\
HQQ & & & \\
KVQuant & & & \\
\bottomrule
\end{tabular}
\end{table}

\subsubsection{Cross-Task Performance Summary}

Table \ref{tab:task_summary} provides consolidated task performance with overall rankings.

\begin{table*}[H]
\centering
\caption{Comprehensive task performance summary}
\label{tab:task_summary}
\footnotesize
\begin{tabular}{lcccccccc}
\toprule
\textbf{Method} & \textbf{Perplexity} & \textbf{Commonsense} & \textbf{Math} & \textbf{Knowledge} & \textbf{Code} & \textbf{Average} & \textbf{Avg Deg.} & \textbf{Quality} \\
 & \textit{(↓)} & \textbf{Avg} & \textbf{(GSM8K)} & \textbf{(MMLU)} & \textbf{(HumanEval)} & \textbf{Accuracy} & \textbf{(\%)} & \textbf{Rank} \\
\midrule
FP16 & 12.79 & 0.687 & 0.360 & 1.000 & 0.050 & 0.524 & 0.0\% & 1 \\
NF4 & & & & & & & & \\
GPTQ & & & & & & & & \\
AWQ & & & & & & & & \\
QuaRot & & & & & & & & \\
HQQ & & & & & & & & \\
KVQuant & & & & & & & & \\
\bottomrule
\end{tabular}
\end{table*}

\subsubsection{Task-Specific Sensitivity Analysis}

Table \ref{tab:sensitivity_analysis} identifies which task categories exhibit highest and lowest sensitivity to quantization.

\begin{table}[H]
\centering
\caption{Task category sensitivity to quantization}
\label{tab:sensitivity_analysis}
\small
\begin{tabular}{lccc}
\toprule
\textbf{Task Category} & \textbf{Mean Degradation} & \textbf{Std Dev} & \textbf{Sensitivity} \\
 & \textbf{(across methods)} & & \textbf{Level} \\
\midrule
Mathematical Reasoning & & & \\
Code Generation & & & \\
Commonsense Reasoning & & & \\
World Knowledge & & & \\
Language Modeling & & & \\
\bottomrule
\end{tabular}
\end{table}

\subsection{RAG Performance Analysis}
\label{sec:rag}

\subsubsection{Answer Generation Quality}

Table \ref{tab:rag_answer_quality} evaluates generated answer quality using multiple complementary metrics.

\begin{table}[H]
\centering
\caption{RAG answer quality metrics}
\label{tab:rag_answer_quality}
\small
\begin{tabular}{lcccccc}
\toprule
\textbf{Method} & \textbf{F1} & \textbf{EM} & \textbf{ROUGE-1} & \textbf{ROUGE-L} & \textbf{BERTScore} & \textbf{Avg} \\
 & & & & & \textbf{F1} & \textbf{Score} \\
\midrule
FP16 & & & & & & \\
NF4 & & & & & & \\
GPTQ & & & & & & \\
AWQ & & & & & & \\
QuaRot & & & & & & \\
HQQ & & & & & & \\
KVQuant & & & & & & \\
\bottomrule
\end{tabular}
\end{table}

\subsubsection{Faithfulness and Relevance}

Table \ref{tab:rag_faithfulness} measures answer adherence to retrieved context and query relevance.

\begin{table}[H]
\centering
\caption{RAG faithfulness and relevance metrics}
\label{tab:rag_faithfulness}
\small
\begin{tabular}{lcccc}
\toprule
\textbf{Method} & \textbf{Faithfulness} & \textbf{Relevance} & \textbf{Avg Answer} & \textbf{Hallucination} \\
 & & & \textbf{Length} & \textbf{Rate} \\
\midrule
FP16 & & & & \\
NF4 & & & & \\
GPTQ & & & & \\
AWQ & & & & \\
QuaRot & & & & \\
HQQ & & & & \\
KVQuant & & & & \\
\bottomrule
\end{tabular}
\end{table}

\subsubsection{Retrieval Quality Assessment}

Table \ref{tab:retrieval_quality} characterizes the quality of context retrieval independent of generation.

\begin{table}[H]
\centering
\caption{Context retrieval quality metrics}
\label{tab:retrieval_quality}
\small
\begin{tabular}{lccccc}
\toprule
\textbf{Method} & \textbf{Context} & \textbf{Context} & \textbf{Context} & \textbf{Avg Score} & \textbf{Avg Context} \\
 & \textbf{Sufficiency} & \textbf{Precision} & \textbf{Coverage} & & \textbf{Length (chars)} \\
\midrule
FP16 & & & & & \\
NF4 & & & & & \\
GPTQ & & & & & \\
AWQ & & & & & \\
QuaRot & & & & & \\
HQQ & & & & & \\
KVQuant & & & & & \\
\bottomrule
\end{tabular}
\end{table}

\subsubsection{RAG vs. No-RAG Comparison}

Table \ref{tab:rag_comparison} quantifies the improvement gained from retrieval augmentation versus direct generation.

\begin{table}[H]
\centering
\caption{RAG improvement over no-RAG baseline}
\label{tab:rag_comparison}
\small
\begin{tabular}{lccccc}
\toprule
\textbf{Method} & \textbf{No-RAG} & \textbf{RAG} & \textbf{F1} & \textbf{Relative} & \textbf{RAG} \\
 & \textbf{F1} & \textbf{F1} & \textbf{Improvement} & \textbf{Improvement} & \textbf{Benefit} \\
\midrule
FP16 & & & & & \\
NF4 & & & & & \\
GPTQ & & & & & \\
AWQ & & & & & \\
QuaRot & & & & & \\
HQQ & & & & & \\
KVQuant & & & & & \\
\bottomrule
\end{tabular}
\end{table}

\subsubsection{RAG Efficiency Metrics}

Table \ref{tab:rag_efficiency} measures computational overhead introduced by retrieval augmentation.

\begin{table}[H]
\centering
\caption{RAG computational efficiency}
\label{tab:rag_efficiency}
\small
\begin{tabular}{lcccccc}
\toprule
\textbf{Method} & \textbf{Retrieval} & \textbf{RAG Gen} & \textbf{No-RAG Gen} & \textbf{RAG} & \textbf{No-RAG} & \textbf{Gen} \\
 & \textbf{Time (ms)} & \textbf{Time (ms)} & \textbf{Time (ms)} & \textbf{Throughput} & \textbf{Throughput} & \textbf{Slowdown} \\
\midrule
FP16 & & & & & & \\
NF4 & & & & & & \\
GPTQ & & & & & & \\
AWQ & & & & & & \\
QuaRot & & & & & & \\
HQQ & & & & & & \\
KVQuant & & & & & & \\
\bottomrule
\end{tabular}
\end{table}

\subsubsection{RAG Performance Summary}

Table \ref{tab:rag_summary} consolidates RAG metrics into overall scores for direct comparison.

\begin{table*}[H]
\centering
\caption{Comprehensive RAG performance summary}
\label{tab:rag_summary}
\small
\begin{tabular}{lcccccccc}
\toprule
\textbf{Method} & \textbf{Answer} & \textbf{Faithful-} & \textbf{Retrieval} & \textbf{RAG} & \textbf{RAG} & \textbf{Overall} & \textbf{RAG Deg.} & \textbf{RAG} \\
 & \textbf{Quality} & \textbf{ness} & \textbf{Quality} & \textbf{Improvement} & \textbf{Efficiency} & \textbf{RAG Score} & \textbf{vs FP16} & \textbf{Rank} \\
\midrule
FP16 & & & & & & & 0.0\% & 1 \\
NF4 & & & & & & & & \\
GPTQ & & & & & & & & \\
AWQ & & & & & & & & \\
QuaRot & & & & & & & & \\
HQQ & & & & & & & & \\
KVQuant & & & & & & & & \\
\bottomrule
\end{tabular}
\end{table*}

\subsection{Comparative Analysis}

\subsubsection{Efficiency-Quality Pareto Frontier}

Table \ref{tab:pareto_analysis} identifies methods that achieve optimal trade-offs between compression efficiency and task performance.

\begin{table}[H]
\centering
\caption{Pareto optimality analysis: efficiency vs. quality}
\label{tab:pareto_analysis}
\small
\begin{tabular}{lcccccc}
\toprule
\textbf{Method} & \textbf{Throughput} & \textbf{Memory} & \textbf{Task} & \textbf{RAG} & \textbf{Pareto} & \textbf{Use} \\
 & \textbf{Rank} & \textbf{Rank} & \textbf{Quality} & \textbf{Quality} & \textbf{Optimal} & \textbf{Case} \\
\midrule
FP16 & & & Baseline & Baseline & & \\
NF4 & & & & & & \\
GPTQ & & & & & & \\
AWQ & & & & & & \\
QuaRot & & & & & & \\
HQQ & & & & & & \\
KVQuant & & & & & & \\
\bottomrule
\end{tabular}
\end{table}

\subsubsection{Method Clustering Analysis}

Table \ref{tab:method_clustering} groups methods by behavioral similarity across all evaluation dimensions.

\begin{table}[H]
\centering
\caption{Method clustering by performance characteristics}
\label{tab:method_clustering}
\small
\begin{tabular}{llp{7cm}}
\toprule
\textbf{Cluster} & \textbf{Methods} & \textbf{Shared Characteristics} \\
\midrule
High Efficiency & & \\
Balanced & & \\
Quality-Focused & & \\
Specialized & & \\
\bottomrule
\end{tabular}
\end{table}

\subsubsection{Hardware Compatibility Summary}

Table \ref{tab:hardware_compatibility} documents observed hardware-specific behaviors on Tesla T4 (Turing architecture).

\begin{table}[H]
\centering
\caption{Hardware compatibility assessment for Tesla T4}
\label{tab:hardware_compatibility}
\small
\begin{tabular}{lccccc}
\toprule
\textbf{Method} & \textbf{Native} & \textbf{Kernel} & \textbf{Observed} & \textbf{T4} & \textbf{Deployment} \\
 & \textbf{Support} & \textbf{Fallback} & \textbf{Issues} & \textbf{Viable} & \textbf{Notes} \\
\midrule
FP16 & & & & & \\
NF4 & & & & & \\
GPTQ & & & & & \\
AWQ & & & & & \\
QuaRot & & & & & \\
HQQ & & & & & \\
KVQuant & & & & & \\
\bottomrule
\end{tabular}
\end{table}

\subsubsection{Deployment Decision Framework}

Table \ref{tab:deployment_recommendations} provides actionable guidance for method selection based on deployment priorities.

\begin{table*}[H]
\centering
\caption{Method selection recommendations by deployment scenario}
\label{tab:deployment_recommendations}
\small
\begin{tabular}{lp{10cm}l}
\toprule
\textbf{Deployment Scenario} & \textbf{Recommended Method} & \textbf{Rationale} \\
\midrule
Maximum Quality Preservation & & \\
Minimum Memory Footprint & & \\
Maximum Throughput & & \\
Long-Context RAG (>4K tokens) & & \\
Hardware-Constrained (Turing GPUs) & & \\
Zero-Shot Deployment & & \\
Balanced Performance & & \\
\bottomrule
\end{tabular}
\end{table*}

\subsection{Key Findings Summary}

Based on the comprehensive evaluation results, we identify several critical insights:

\begin{table}[H]
\centering
\caption{Summary of key findings}
\label{tab:key_findings}
\small
\begin{tabular}{lp{10cm}}
\toprule
\textbf{Finding Category} & \textbf{Key Insight} \\
\midrule
Best Overall Method & \\
Lowest Degradation & \\
Highest Efficiency & \\
Best for RAG & \\
Best for Math & \\
Best for T4 Hardware & \\
Task Sensitivity Ranking & \\
Compression-Quality Sweet Spot & \\
\bottomrule
\end{tabular}
\end{table}