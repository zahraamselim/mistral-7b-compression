\documentclass[11pt,a4paper]{article}
% \documentclass[conference]{IEEEtran}
% \IEEEoverridecommandlockouts

% Packages
\usepackage{cite}
\usepackage{amsmath,amssymb,amsfonts}
\usepackage{algorithm}
\usepackage{algorithmic}
\usepackage{graphicx}
\usepackage{textcomp}
\usepackage{xcolor}
\usepackage{booktabs}
\usepackage{multirow}
\usepackage{float}
\usepackage{listings}
\usepackage{url}
\usepackage{enumitem}
\usepackage{titlesec}
\usepackage{adjustbox}
\usepackage[utf8]{inputenc}
\usepackage[T1]{fontenc}
\usepackage{lmodern}
\usepackage[margin=1in]{geometry}

% Column spacing - more breathing room between columns
\setlength{\columnsep}{0.25in}
\setlength{\parindent}{0pt}

% Table spacing - makes tables more readable
\renewcommand{\arraystretch}{1.3}

% Section spacing - better vertical rhythm
\titlespacing*{\section}{0pt}{12pt plus 4pt minus 2pt}{6pt plus 2pt minus 2pt}
\titlespacing*{\subsection}{0pt}{10pt plus 3pt minus 2pt}{4pt plus 2pt minus 2pt}
\titlespacing*{\subsubsection}{0pt}{8pt plus 2pt minus 2pt}{3pt plus 2pt minus 2pt}

% Paragraph spacing
\setlength{\parskip}{3pt}

% List spacing - cleaner itemize/enumerate
\setlist{itemsep=2pt, parsep=2pt, topsep=4pt}

% Code listing configuration
\lstset{
  language=Python,
  basicstyle=\ttfamily\scriptsize,
  numbers=left,
  numberstyle=\tiny\color{gray},
  keywordstyle=\color{blue},
  commentstyle=\color{green!50!black},
  stringstyle=\color{red!70!black},
  showstringspaces=false,
  breaklines=true,
  frame=single,
  aboveskip=10pt,
  belowskip=10pt,
  xleftmargin=15pt,
  xrightmargin=5pt,
  framexleftmargin=12pt
}

% Better spacing around equations
\setlength{\abovedisplayskip}{8pt plus 2pt minus 4pt}
\setlength{\belowdisplayskip}{8pt plus 2pt minus 4pt}
\setlength{\abovedisplayshortskip}{4pt plus 2pt minus 2pt}
\setlength{\belowdisplayshortskip}{4pt plus 2pt minus 2pt}

% BibTeX definition
\def\BibTeX{{\rm B\kern-.05em{\sc i\kern-.025em b}\kern-.08em
    T\kern-.1667em\lower.7ex\hbox{E}\kern-.125emX}}

\begin{document}

% \title{Comprehensive Evaluation of Quantization Methods for Edge LLM Deployment\\
% \thanks{This work was conducted at Egypt-Japan University of Science and Technology (E-JUST).}
% }

% \author{\IEEEauthorblockN{Zahraa Selim, Menna Hamed, Wesam Ahmed, Sohyla Said, Sara Basheer, and Rami Zewail}
% \IEEEauthorblockA{\textit{Computer Science and Engineering Department}\\
% \textit{Egypt-Japan University of Science and Technology (E-JUST)}\\
% Alexandria, Egypt\\
% \{zahraa.selim, menna.hamed, rami.zewail\}@ejust.edu.eg}
% }

\title{\LARGE \textbf{Comprehensive Evaluation of Quantization Methods\\for Edge LLM Deployment with RAG Focus}}

\author{
Zahraa Selim, Menna Hamed, Wesam Ahmed,\\
Sohyla Said, Sara Basheer, Rami Zewail\\
\\
\textit{Computer Science and Engineering Department}\\
\textit{Egypt-Japan University of Science and Technology (E-JUST)}\\
Alexandria, Egypt\\
\texttt{\{zahraa.selim, menna.hamed, rami.zewail\}@ejust.edu.eg}
}

\maketitle

\begin{abstract}
The deployment of Large Language Models (LLMs) on resource-constrained edge hardware is significantly impeded by memory bandwidth bottlenecks. Post-Training Quantization (PTQ) has emerged as a standard compression paradigm, yet comprehensive benchmarking of modern quantization methods on consumer-grade hardware remains limited. This study conducts a rigorous empirical evaluation of five advanced quantization techniques (NF4, GPTQ, AWQ, HQQ, and AutoRound) applied to Mistral-7B on NVIDIA Tesla T4 hardware. Our findings indicate that NormalFloat 4-bit (NF4) quantization establishes an optimal Pareto frontier, achieving 3.6× memory reduction while maintaining superior task performance and hardware compatibility. Critically, we identify that hardware-algorithm compatibility significantly influences performance, with distribution-based methods (NF4) providing superior stability on Turing architectures compared to kernel-dependent methods (GPTQ, HQQ) that may experience severe fallback overhead. Our analysis reveals task-dependent sensitivity to quantization, where mathematical reasoning degrades by 25\% while knowledge retrieval remains robust. We provide prescriptive deployment guidelines identifying NF4 as the superior strategy for balancing throughput, energy efficiency, and task performance on edge hardware.
\end{abstract}

% \begin{IEEEkeywords}
% LLM Quantization, Edge Computing, Mistral-7B, Efficiency Benchmarking, Post-Training Quantization, Hardware Compatibility
% \end{IEEEkeywords}

% Updated introduction.tex to ensure consistency with six methods and emphasize RAG focus with novel metrics
\section{Introduction}

The democratization of Large Language Models (LLMs) represents a transformative shift in artificial intelligence accessibility. Models such as Mistral-7B \cite{jiang2023mistral}, with their balance of capability and efficiency, exemplify the potential for deploying sophisticated reasoning systems beyond cloud infrastructure. However, the transition from datacenter-grade hardware to consumer and edge devices—such as the NVIDIA T4 and RTX series GPUs—confronts a fundamental bottleneck: memory bandwidth and capacity constraints.

A typical 7-billion parameter model in half-precision (FP16) format requires approximately 14GB of Video Random Access Memory (VRAM) for weights alone, consuming nearly the entire capacity of a 16GB GPU and leaving minimal headroom for the dynamic memory allocations essential during inference. The Key-Value (KV) cache, which grows linearly with sequence length, further exacerbates this constraint. For a context window of 4096 tokens, the KV cache alone demands an additional ~1GB of memory, making longer-context applications infeasible without compression.

Post-Training Quantization (PTQ) techniques offer a compelling solution by reducing numerical precision without the computational overhead of retraining. Methods such as GPTQ \cite{frantar2023gptq}, AWQ \cite{lin2023awq}, and NormalFloat 4-bit (NF4) \cite{dettmers2023qlora} theoretically achieve 3.5-4$\times$ compression ratios, reducing model footprints to approximately 3.5-4GB. However, quantization introduces non-uniform degradation patterns across different capabilities—a phenomenon inadequately captured by aggregate metrics like perplexity or average benchmark scores.

\subsection{Research Gaps and Motivation}

Current quantization literature exhibits three critical limitations that motivate this investigation:

\paragraph{Incomplete Understanding of Task-Specific Sensitivity}

Existing evaluations predominantly rely on perplexity measurements and broad benchmark suites (e.g., MMLU, HellaSwag), which provide aggregate performance indicators but obscure differential sensitivities across task categories. Mathematical reasoning, for instance, may degrade more severely than factual recall due to quantization-induced precision loss in arithmetic operations. Code generation may suffer disproportionately from syntactic errors that small quantization perturbations introduce. Without granular, capability-specific evaluation, practitioners lack guidance on which compression methods best preserve the capabilities most relevant to their deployment scenarios.

\paragraph{Hardware-Software Co-optimization Neglect}

Quantization methods are often evaluated in hardware-agnostic settings, assuming universal kernel support and optimal implementations. This assumption breaks down in practice: quantization schemes optimized for Ampere architecture (e.g., INT4 tensor cores) may experience severe performance degradation on older Turing GPUs (e.g., Tesla T4), which lack native low-precision arithmetic support and rely on software emulation. The interaction between quantization algorithm design and actual hardware capabilities remains underexplored, yet critically determines real-world deployment viability.

\paragraph{Limited Retrieval-Augmented Generation (RAG) Analysis}

RAG systems, which augment language models with external knowledge retrieval, represent a dominant deployment paradigm for domain-specific applications. However, quantization's impact on RAG workflows—specifically on context adherence, multi-hop reasoning across retrieved documents, and faithfulness to source material—remains largely uncharacterized. The compound effects of compressing both the retrieval mechanism (through embedding quantization) and the generation model (through weight quantization) require systematic investigation, particularly through novel metrics like attention preservation, context degradation, and attention drift.

\subsection{Research Contributions}

This work addresses these gaps through a comprehensive evaluation framework that examines six state-of-the-art quantization methods across three analytical dimensions:

\paragraph{Diverse Quantization Paradigms} We evaluate methods spanning fundamentally different compression principles:
\begin{itemize}
    \item \textbf{Data-type innovations (NF4)}: Information-theoretically optimal representations for normal distributions
    \item \textbf{Hessian-based optimization (GPTQ)}: Second-order curvature-aware quantization minimizing reconstruction error
    \item \textbf{Activation-aware weighting (AWQ)}: Salience-based protection of critical weights identified through activation magnitudes
    \item \textbf{Rotation-based outlier redistribution (QuaRot)}: Hadamard transformations to smooth activation distributions
    \item \textbf{Statistical adaptivity (HQQ)}: Zero-shot quantization without calibration data
    \item \textbf{Memory hierarchy optimization (KVQuant)}: Specialized compression for attention KV caches
\end{itemize}

This breadth enables systematic comparison of orthogonal design philosophies rather than incremental algorithmic variations.

\paragraph{Multi-Dimensional Evaluation Protocol} We assess compressed models across:
\begin{itemize}
    \item \textbf{Computational efficiency}: Latency (TTFT, decode time), throughput, memory footprint, energy consumption, and Model FLOPs Utilization (MFU)
    \item \textbf{Task-specific performance}: Granular evaluation across mathematical reasoning (GSM8K), commonsense reasoning (HellaSwag, ARC), world knowledge (MMLU), and code generation (HumanEval)
    \item \textbf{RAG capabilities}: Novel metrics including attention preservation (precision@1 and rank of relevant documents), context degradation (accuracy slope over increasing lengths), and attention drift (stability during generation)
\end{itemize}

\paragraph{Hardware-Aware Deployment Analysis} All experiments are conducted on a single NVIDIA Tesla T4 GPU (16GB VRAM), representative of edge deployment scenarios. This constraint reveals critical insights about algorithm-hardware compatibility: methods requiring specialized kernel support (e.g., structured sparsity, INT4 tensor cores) may underperform distribution-based approaches (e.g., NF4) that gracefully degrade to software implementations on older architectures.

Our findings demonstrate that quantization method selection must account for deployment context. For mathematical reasoning preservation, Hessian-based methods (GPTQ) minimize precision-sensitive degradation. For memory-constrained RAG applications, KV cache quantization (KVQuant) provides orthogonal compression benefits beyond weight quantization alone. For hardware compatibility on Turing-generation GPUs, distribution-based methods (NF4, HQQ) avoid kernel fallback penalties that plague methods assuming Ampere-specific optimizations.

\subsection{Paper Organization}

The remainder of this paper proceeds as follows: Section II surveys related work in quantization methods, organizing techniques by underlying compression principles rather than chronological development. Section III details our experimental methodology, including quantization configurations, evaluation protocols, and RAG pipeline design with novel attention-based metrics. Section IV presents comprehensive results across efficiency, quality, and RAG dimensions. Section V discusses implications for deployment strategy and identifies optimal method selection criteria. Section VI concludes with future research directions.

% Updated related_work.tex - no major changes needed for coherence, as it already covers the methods consistently
\section{Related Work}

Quantization for large language models has emerged as a critical research area, with techniques spanning diverse compression principles. We organize this survey by underlying methodology rather than chronological development, highlighting how different approaches address the fundamental trade-off between compression ratio and capability preservation.

\subsection{Curvature-Based Optimization Methods}

These methods leverage second-order information about the loss surface to identify optimal quantization parameters that minimize reconstruction error.

\paragraph{Hessian-Guided Quantization}

\textbf{GPTQ} \cite{frantar2023gptq} pioneered practical Hessian-based quantization for billion-parameter models through layer-wise optimization with inverse Hessian approximations. The method formulates quantization as minimizing $\|WX - \hat{W}X\|_2^2$ where $W$ is the original weight matrix and $\hat{W}$ is quantized, using a greedy coordinate descent approach that propagates quantization errors to subsequent weights via $w_{q'} \leftarrow w_{q'} - \delta_q \cdot \frac{H_{qq'}}{H_{qq}}$. This Optimal Brain Quantization (OBQ) strategy enables 3-4 bit quantization with minimal perplexity degradation.

\textbf{QuIP} \cite{chee2023quip} extends this framework to 2-bit precision through adaptive rounding and incoherence processing, applying randomized Hadamard transforms before quantization to reduce weight correlations. \textbf{QuIP\#} further improves this approach with fine-tuned incoherence preprocessing, achieving state-of-the-art 2-bit performance.

\textbf{OPTQ} variants including \textbf{GPTAQ} and \textbf{MR-GPTQ} introduce mixed-precision strategies, allocating higher precision to layers identified as sensitive through Hessian eigenvalue analysis. \textbf{OPQ} (Outlier-Preserved Quantization) combines Hessian guidance with explicit outlier handling, storing high-magnitude weights in elevated precision.

\textbf{AutoRound} employs iterative reconstruction with adaptive rounding, optimizing quantization parameters per-layer through gradient descent on reconstruction error. Unlike GPTQ's one-shot approach, AutoRound refines parameters across multiple iterations, often improving upon GPTQ with minimal computational overhead.

\paragraph{Curvature Approximations}

Several methods approximate Hessian computation to reduce overhead. \textbf{QUAD} uses diagonal Hessian approximations, trading accuracy for speed. \textbf{FPTQ} (Fast Post-Training Quantization) employs block-wise Hessian estimation, processing weight submatrices independently. \textbf{ResQ} introduces residual quantization, iteratively quantizing reconstruction errors to achieve finer effective precision.

\subsection{Activation-Aware and Salience-Based Methods}

These approaches identify and protect critical weights based on activation statistics or gradient information, recognizing that uniform quantization degrades performance by treating all parameters equally.

\paragraph{Salience-Based Weight Protection}

\textbf{AWQ} (Activation-aware Weight Quantization) \cite{lin2023awq} observes that 1\% of weights disproportionately impact model outputs when weighted by activation magnitudes. By applying per-channel scaling factors $s = (\text{mean}(|X|_{\alpha}))^{\alpha}$ where $X$ represents activations, AWQ protects salient weights while aggressively quantizing less-critical parameters. This salience-based approach demonstrates superior preservation of reasoning capabilities compared to reconstruction-based methods.

\textbf{OWQ} (Outlier-Aware Weight Quantization) extends this principle by explicitly identifying weights sensitive to activation outliers, applying mixed-precision storage where sensitivity exceeds thresholds. \textbf{SpQR} (Sparse-Quantized Representation) uses L2 error as a sensitivity metric, storing the top-k sensitive weights in higher precision while quantizing the remainder.

\textbf{SqueezeLLM} \cite{kim2023squeezellm} introduces sensitivity-based weight clustering via k-means, grouping weights by importance scores and allocating precision budgets per cluster. This enables non-uniform quantization that concentrates bits where they most impact model quality.

\textbf{WUSH} (Weight-Update Salience Heuristic) tracks weight update magnitudes during fine-tuning as a proxy for importance, protecting high-update weights during subsequent quantization. \textbf{AdpQ} (Adaptive Precision Quantization) dynamically adjusts per-layer precision based on activation variance, allocating more bits to layers with high variance.

\paragraph{Gradient-Based Importance}

\textbf{XQuant} leverages gradient information to identify critical weight columns, applying column-wise mixed precision. \textbf{GQSA} (Gradient-Quantization Sensitivity Analysis) computes sensitivity scores as $\nabla_W \mathcal{L}$, protecting high-gradient regions. \textbf{EWQ} (Error-Weighted Quantization) uses backpropagated errors to guide precision allocation, iteratively refining quantization to minimize task-specific loss.

\subsection{Outlier Redistribution and Smoothing}

Activation outliers—extreme values that occur in specific channels—pose significant challenges for quantization, as they force large quantization ranges that waste representation capacity. These methods address outliers through transformation or redistribution rather than preservation.

\paragraph{Rotation-Based Approaches}

\textbf{QuaRot} applies fixed Hadamard rotation matrices $H$ to weight matrices before quantization, transforming $W \rightarrow HWH^T$, which redistributes outliers across channels while preserving the mathematical equivalence through inverse rotations during inference. This approach achieves 4-bit quantization competitive with 16-bit baselines by homogenizing activation distributions.

\textbf{SpinQuant} extends rotation methods with learnable orthogonal transformations optimized to maximize quantization friendliness, formulated as $\min_R \|Q(RW) - RW\|$ subject to $RR^T = I$. By learning rotation matrices rather than using fixed transforms, SpinQuant adapts to model-specific outlier patterns.

\textbf{ButterflyQuant} employs structured butterfly matrices for efficient rotation, reducing the $O(n^2)$ rotation cost to $O(n \log n)$ while maintaining outlier smoothing effectiveness.

\paragraph{Channel-Wise Smoothing}

\textbf{SmoothQuant} \cite{xiao2023smoothquant} migrates quantization difficulty from activations to weights through per-channel scaling: $Y = (X \text{diag}(s)^{-1}) \cdot (\text{diag}(s)W)$, where scaling factors $s$ balance difficulty. This reduces outlier impact without architectural changes.

\textbf{Outlier Suppression+} \cite{wei2022outlier} suppresses outliers through activation clipping and shifting. \textbf{CliP} combines clipping with progressive quantization, gradually reducing precision while monitoring error.

\textbf{RepQ} (Representation Quantization) normalizes activations to unit variance before quantization, redistributing dynamic range. \textbf{Outlier Token Suppression} identifies and suppresses outlier-causing tokens during inference.

\subsection{Mixed-Precision and Heterogeneous Quantization}

These methods allocate variable precision across model components, recognizing that uniform bit-width sacrifices efficiency.

\paragraph{Layer-Wise Precision Allocation}

\textbf{MixPrecision} dynamically assigns precision based on layer sensitivity, measured by signal-to-quantization-noise ratio (SQNR). Layers with high SQNR tolerate lower precision.

\textbf{LLM.int8()} \cite{dettmers2022llmint8} uses INT8 for regular weights and FP16 for outlier features identified by vector norms exceeding thresholds.

\textbf{FlexiQuant} allows arbitrary bit-width combinations per layer, optimized via search algorithms. \textbf{MPQ} (Multi-Precision Quantization) groups layers by type (attention vs. feed-forward) for precision assignment.

\paragraph{Component-Specific Strategies}

\textbf{Token-wise Quantization} applies per-token scaling, adapting to varying activation distributions across sequences. \textbf{Group-wise Quantization} partitions weights into groups with shared scaling factors, balancing granularity and overhead.

\textbf{CQ} (Channel Quantization) applies different precision to different attention head groups, allocating higher bits to heads responsible for long-range dependencies.

\subsection{KV Cache Compression}

The Key-Value cache in Transformer attention grows linearly with sequence length, dominating memory consumption for long-context applications. KV cache quantization provides orthogonal compression beyond weight quantization.

\paragraph{Per-Channel and Per-Token Strategies}

\textbf{KVQuant} introduces separate quantization strategies for keys and values: per-channel quantization for keys (which remain relatively stable across tokens) and per-token quantization for values (which vary significantly). Additionally, KVQuant preserves \textit{attention sinks}—initial tokens that accumulate disproportionate attention—in full precision to maintain attention distribution fidelity.

\textbf{KIVI} (KV Cache In-place Inference) implements 2-bit KV cache quantization with asymmetric quantization schemes, achieving 4$\times$ KV cache compression. By quantizing in-place during generation, KIVI minimizes memory overhead and enables million-token contexts on consumer hardware.

\textbf{AsymKV} applies asymmetric quantization to keys and values separately, recognizing their different statistical properties. \textbf{AQUA-KV} (Adaptive Quantization for KV Cache) adjusts quantization granularity based on sequence length, using coarse quantization for long contexts and fine quantization for short contexts.

\paragraph{Attention-Aware KV Compression}

\textbf{IntactKV} preserves high-attention keys and values in elevated precision, dynamically identifying important cache entries based on cumulative attention scores. \textbf{WKVQuant} applies cross-block reconstruction regularization, ensuring that quantization errors in one layer's KV cache don't compound in subsequent layers.

\textbf{TEQ} (Token-Efficient Quantization) combines KV cache quantization with token pruning, removing low-attention tokens from the cache entirely.

\subsection{Quantization-Aware Training and Fine-Tuning}

While post-training quantization requires no retraining, quantization-aware training (QAT) can further reduce degradation by adapting model parameters to low-precision representations.

\paragraph{Full QAT Methods}

\textbf{LLM-QAT} implements standard QAT through knowledge distillation from full-precision teachers, training quantized models to mimic teacher outputs. \textbf{EfficientQAT} reduces QAT cost through progressive quantization, starting with high precision and gradually reducing bit-widths during training.

\textbf{BitNet} and \textbf{BiLLM} explore 1-bit weight quantization with QAT, representing weights as $\{-1, +1\}$. \textbf{PB-LLM} (Partially Binarized LLM) applies 1-bit quantization selectively, binarizing only low-variance weight blocks.

\paragraph{Parameter-Efficient QAT}

\textbf{QLoRA} \cite{dettmers2023qlora} combines NF4 quantization with Low-Rank Adaptation (LoRA), training low-rank adapters on top of frozen quantized weights. This enables fine-tuning 65B models on single GPUs by limiting trainable parameters to <1\% of total model size.

\textbf{QA-LoRA} extends this with quantization-aware LoRA initialization, optimizing adapter ranks and placement based on quantization sensitivity. \textbf{LoftQ} iteratively refines LoRA decomposition and quantization parameters jointly. \textbf{PRILoRA} introduces precision-incremental LoRA, gradually reducing precision during adaptation. \textbf{IR-QLoRA} combines importance reweighting with QLoRA for improved preservation of critical capabilities.

\textbf{PEQA} (Parameter-Efficient Quantization Adaptation) applies adapter modules specifically at quantization-sensitive layers, concentrating trainable parameters where they most impact quality recovery.

\subsection{Specialized Domains and Formats}

\paragraph{Code-Specific Quantization}

\textbf{EETQ} (Efficient Embedding and Tokenizer Quantization) addresses quantization for code models, where exact token matching is critical. \textbf{MoFQ} (Mode-Focused Quantization) recognizes that code token distributions are multimodal and designs quantization schemes respecting mode boundaries.

\paragraph{Tensor Format Innovations}

\textbf{GGUF} formats (including \textbf{GGUF-Q} and \textbf{GGUF-IQ}) provide portable quantized model representations supporting mixed precision and efficient memory-mapped inference. These formats enable deployment across diverse hardware without recompilation.

\textbf{torchao} and \textbf{Quanto} provide PyTorch-native quantization APIs, integrating seamlessly with existing training pipelines. \textbf{QServe} offers optimized serving infrastructure for quantized models with dynamic batching and request scheduling.

\subsection{Emerging Directions}

Recent work explores frontiers including:

\textbf{Ultra-Low Precision}: \textbf{INT2.1}, \textbf{ZeroQuant-4+2}, and \textbf{VPTQ} (Variable Precision Tensor Quantization) push toward sub-2-bit quantization through aggressive outlier handling and specialized data types.

\textbf{Learnable Quantization Functions}: \textbf{SGD Weight Rounding}, \textbf{MagR} (Magnitude-Aware Rounding), and \textbf{FlexRound} optimize rounding functions rather than using fixed round-to-nearest.

\textbf{Norm-Based Methods}: \textbf{Norm Tweaking}, \textbf{FrameQuant}, and \textbf{AffineQuant} manipulate layer normalization statistics to create quantization-friendly activation distributions.

\textbf{Distribution Matching}: \textbf{KurTail} matches quantized weight distributions to target kurtosis values, \textbf{DecoupleQ} separates magnitude and sign quantization, \textbf{DuQuant} applies dual-domain quantization in both spatial and frequency domains.

\subsection{Summary and Positioning}

This survey reveals that quantization research has progressed from uniform precision reduction to sophisticated, heterogeneous compression strategies that account for weight salience, activation statistics, hardware constraints, and task-specific requirements. However, systematic comparison across diverse quantization paradigms under controlled conditions remains limited, particularly for RAG applications where context-dependence introduces additional complexity.

Our work fills this gap by evaluating six representative methods spanning orthogonal design principles (NF4, GPTQ, AWQ, QuaRot, HQQ, KVQuant) under unified experimental conditions on standardized hardware (Tesla T4), with particular attention to RAG scenarios underexplored in prior literature. We use standard calibration datasets for all methods requiring calibration (e.g., GPTQ, AWQ), without task-specific or RAG-optimized calibration.

\section{Methodology}

\subsection{Overview}

This study presents a comprehensive Retrieval-Augmented Generation (RAG) system optimized for deployment on consumer-grade hardware using 4-bit quantized Large Language Models. Our modular architecture addresses the compound resource constraints of edge deployment: concurrent allocation for vector indices, embedding models, dynamic Key-Value caches, and the generation model itself. All experiments are conducted on NVIDIA Tesla T4 (16GB VRAM) to simulate realistic edge deployment scenarios.

The RAG pipeline consists of seven sequential modules: (1) Ingestion for document extraction, (2) Processing for content structuring, (3) Chunking for optimal segmentation, (4) Embedding for vector representation, (5) Indexing for fast retrieval, (6) Retrieval for relevant context selection, and (7) Generation for answer synthesis. We evaluate the complete system under quantized model constraints, measuring both efficiency and faithfulness metrics.

\subsection{Hardware and Software Environment}

\textbf{Hardware Configuration:}
\begin{itemize}
    \item \textbf{Platform:} NVIDIA Tesla T4 GPU (16GB VRAM, Turing architecture)
    \item \textbf{TDP:} 70W (for energy consumption calculations)
    \item \textbf{System RAM:} 16GB DDR4
    \item \textbf{Storage:} NVMe SSD for index persistence
\end{itemize}

\textbf{Software Stack:}
\begin{itemize}
    \item \textbf{Framework:} PyTorch 2.x with CUDA 11.8
    \item \textbf{Quantization:} BitsAndBytes (NF4 4-bit), AutoAWQ
    \item \textbf{Embeddings:} sentence-transformers (BGE-small-en-v1.5)
    \item \textbf{Vector Store:} FAISS (IndexHNSWFlat)
    \item \textbf{LLM Inference:} llama-cpp-python (GGUF format)
    \item \textbf{Language Model:} Mistral-7B-Instruct-v0.2 (4-bit quantized)
\end{itemize}

\subsection{Base Models}

\subsubsection{Generation Model: Mistral-7B-Instruct-v0.2}

We select Mistral-7B-Instruct-v0.2 \cite{jiang2023mistral} for answer generation due to its strong instruction-following capabilities and efficient architecture:

\begin{itemize}
    \item \textbf{Parameters:} 7.24 billion
    \item \textbf{Architecture:} Decoder-only transformer with Grouped Query Attention (GQA, 8 KV heads) and Sliding Window Attention (4096-token window)
    \item \textbf{Context Window:} 32k tokens (using 4k for memory safety)
    \item \textbf{Quantization:} 4-bit NF4 via BitsAndBytes
    \item \textbf{Model Size:} 7.6 GB (quantized) vs. 14 GB (FP16)
    \item \textbf{Format:} GGUF Q4\_K\_M for llama-cpp-python inference
\end{itemize}

\textbf{Quantization Configuration:}
\begin{itemize}
    \item \textbf{Method:} 4-bit NormalFloat (NF4) following QLoRA protocol \cite{dettmers2023qlora}
    \item \textbf{Double Quantization:} Enabled (quantizes quantization constants using FP8)
    \item \textbf{Compute Dtype:} FP16 for dequantization operations
    \item \textbf{Block Size:} 64 (default BitsAndBytes configuration)
\end{itemize}

\textbf{Generation Parameters:}
\begin{itemize}
    \item \textbf{Max New Tokens:} 128 (concise answers)
    \item \textbf{Temperature:} 0.3 (low for factuality)
    \item \textbf{Top-p:} 0.9 (nucleus sampling)
    \item \textbf{Repetition Penalty:} 1.15
    \item \textbf{Do Sample:} False (deterministic decoding)
\end{itemize}

\subsubsection{Embedding Model: BGE-small-en-v1.5}

For dense semantic embeddings, we employ BAAI/bge-small-en-v1.5 \cite{xiao2023bge}, a compact sentence transformer optimized for retrieval:

\begin{itemize}
    \item \textbf{Dimensions:} 384 (compact yet expressive)
    \item \textbf{Model Size:} 133 MB on disk, ~300 MB VRAM when loaded
    \item \textbf{Max Sequence Length:} 512 tokens
    \item \textbf{Training:} 1B+ sentence pairs from diverse domains
    \item \textbf{Performance:} SBERT benchmark score: 68.06
    \item \textbf{Inference Speed:} 50ms/chunk on GPU, 200ms/chunk on CPU
    \item \textbf{Normalization:} L2-normalized outputs for cosine similarity via dot product
\end{itemize}

\textbf{Content-Type Prefixes:}

To improve retrieval accuracy, we apply type-specific prefixes \cite{xiao2023bge}:
\begin{itemize}
    \item Math equations: \texttt{"equation: \{content\}"}
    \item Code blocks: \texttt{"code: \{content\}"}
    \item Tables: \texttt{"table: \{content\}"}
    \item Plain text: \texttt{"passage: \{content\}"}
    \item Headings: \texttt{"title: \{content\}"}
\end{itemize}

\subsection{RAG Pipeline Architecture}

Our modular RAG system consists of seven sequential stages, each optimized for edge deployment constraints:

\begin{figure}[H]
\centering
\begin{tikzpicture}[
    node distance=0.8cm,
    block/.style={rectangle, draw, fill=blue!10, text width=2.5cm, text centered, minimum height=0.7cm, font=\small},
    arrow/.style={->, >=stealth, thick}
]

% Main pipeline
\node[block] (ingest) {Ingestion};
\node[block, below of=ingest] (process) {Processing};
\node[block, below of=process] (chunk) {Chunking};
\node[block, below of=chunk] (embed) {Embedding};
\node[block, below of=embed] (index) {Indexing};
\node[block, below of=index] (retrieve) {Retrieval};
\node[block, below of=retrieve] (generate) {Generation};

% Arrows
\draw[arrow] (ingest) -- (process);
\draw[arrow] (process) -- (chunk);
\draw[arrow] (chunk) -- (embed);
\draw[arrow] (embed) -- (index);
\draw[arrow] (index) -- (retrieve);
\draw[arrow] (retrieve) -- (generate);

% Labels
\node[right=0.5cm of ingest, font=\footnotesize, text width=3cm] {PDF $\rightarrow$ Structured Text};
\node[right=0.5cm of process, font=\footnotesize, text width=3cm] {Classification + Enrichment};
\node[right=0.5cm of chunk, font=\footnotesize, text width=3cm] {Semantic Segmentation};
\node[right=0.5cm of embed, font=\footnotesize, text width=3cm] {Dense + Sparse Vectors};
\node[right=0.5cm of index, font=\footnotesize, text width=3cm] {FAISS + BM25 + Graph};
\node[right=0.5cm of retrieve, font=\footnotesize, text width=3cm] {Hybrid Search + RRF};
\node[right=0.5cm of generate, font=\footnotesize, text width=3cm] {Quantized LLM};

\end{tikzpicture}
\caption{Modular RAG pipeline architecture for edge deployment}
\label{fig:rag_pipeline}
\end{figure}

\subsubsection{Module 1: Ingestion}

The ingestion module transforms raw PDF documents into structured, searchable content using intelligent routing to optimize for both speed and quality.

\textbf{Routing Strategy:}

We employ a hybrid extraction approach that analyzes PDF characteristics to select optimal parsers:

\begin{itemize}
    \item \textbf{Native Extraction (70-80\% of pages):} For clean PDFs with text layers using PyMuPDF (0.1-0.3s/page, ~100MB RAM)
    \item \textbf{Layout-Aware Extraction (15-20\%):} For complex multi-column layouts using Marker with Surya layout detection (2-4s/page, ~600MB RAM)
    \item \textbf{OCR-Based Extraction (5-10\%):} Three-stage cascade (Tesseract $\rightarrow$ EasyOCR $\rightarrow$ PaddleOCR) for scanned documents (5-15s/page)
\end{itemize}

\textbf{Quality Validation:}

Each extraction is validated using coherence metrics:
\begin{equation}
\text{Coherence} = \frac{\text{Valid\_Characters}}{\text{Total\_Characters}} \times \frac{\text{Sentence\_Count}}{\text{Expected\_Sentences}}
\end{equation}

Extractions scoring below 0.5 are retried with stronger methods (maximum 3 attempts).

\textbf{Specialized Extractors:}

\begin{itemize}
    \item \textbf{Equations:} Multi-method LaTeX extraction (pattern matching, embedded LaTeX, optional Pix2Tex) with SymPy enrichment for variable extraction and natural language descriptions
    \item \textbf{Code Blocks:} Detection via markdown syntax, indentation patterns, and keyword density with language identification for 12+ languages
    \item \textbf{Tables:} Extraction with pdfplumber preserving structure, header detection, and cell alignment
    \item \textbf{Images:} Extraction with bounding boxes, caption detection using proximity heuristics, optional OCR for text-containing images
\end{itemize}

\subsubsection{Module 2: Processing}

The processing module transforms extracted content into semantically rich, type-classified blocks.

\textbf{Content Classification:}

Heuristic-based classification (1-3ms/block) assigns types without ML models:

\begin{itemize}
    \item \textbf{Math:} LaTeX delimiters, mathematical operators ($\int, \sum, \frac{}{}$)
    \item \textbf{Code:} Function definitions, imports, keywords (\texttt{def, class, function})
    \item \textbf{Table:} Pipe separators, column alignment patterns
    \item \textbf{List:} Bullet points, numbered items
    \item \textbf{Heading:} Short capitalized lines, section numbers, markdown headers
    \item \textbf{Text:} Default for standard prose
\end{itemize}

Each type receives a confidence score (0-1), with classification threshold at 0.6.

\textbf{Hierarchy Building:}

Document structure is detected via:
\begin{itemize}
    \item Markdown headers (\texttt{\#, \#\#, \#\#\#})
    \item Numbered sections (1., 1.1, 1.1.1)
    \item Chapter patterns (Chapter 1, CHAPTER I)
    \item Implicit capitalized headings
\end{itemize}

The resulting tree structure enables section-aware retrieval and hierarchical chunking.

\textbf{Relationship Mapping:}

Semantic links are established between blocks:
\begin{itemize}
    \item \textbf{Adjacent:} Sequential blocks (confidence: 1.0)
    \item \textbf{References:} Explicit mentions (Figure N, Table N, Equation N) (confidence: 0.8)
    \item \textbf{Explained\_by:} Math equations with nearby explanatory text (confidence: 0.7)
    \item \textbf{Described\_by:} Code blocks with surrounding descriptions (confidence: 0.7)
\end{itemize}

\textbf{Type-Specific Enrichment:}

\begin{itemize}
    \item \textbf{Math:} SymPy normalization, variable extraction, complexity indicators (calculus, algebra, matrices), natural language descriptions
    \item \textbf{Code:} Language detection, function/class extraction, import analysis, structure summary
    \item \textbf{Tables:} Row/column counting, header detection, type classification (comparison, financial, performance), summary generation
    \item \textbf{Text:} Definition extraction, acronym detection, key term identification (top 10 capitalized phrases), word/sentence counting
\end{itemize}

\textbf{Normalization:}

Text cleaning removes PDF artifacts:
\begin{itemize}
    \item Whitespace collapse and hyphenation rejoining
    \item Quote standardization (curly $\rightarrow$ straight)
    \item Header/footer removal (heuristic: short lines with "page"/numbers)
    \item Unit expansion (km $\rightarrow$ kilometers)
    \item Scientific notation conversion ($3.5 \times 10^2 \rightarrow 350.0$)
\end{itemize}

\subsubsection{Module 3: Chunking}

The chunking module creates optimized segments preserving semantic coherence and special content integrity.

\textbf{Fixed-Smart Chunking (Default):}

Our primary strategy targets 512 tokens (min 256, max 768) with 50-token overlap:

\begin{enumerate}
    \item Accumulate blocks until target size reached
    \item Check if next block exceeds maximum
    \item Finalize chunk at smart boundary (paragraph $>$ sentence $>$ clause)
    \item Add 50-token overlap from previous chunk
    \item Continue with remaining content
\end{enumerate}

\textbf{Boundary Detection:}

Content-aware boundary detector identifies safe split points:
\begin{itemize}
    \item Section breaks and paragraph boundaries (highest priority)
    \item Sentence endings (. ! ?)
    \item Clause boundaries (, ; :)
    \item Before/after special content blocks
\end{itemize}

\textbf{Never splits:}
\begin{itemize}
    \item Inside equations (\texttt{\$...\$, \$\$...\$\$})
    \item Inside code blocks (\texttt{```...```})
    \item Inside tables (\texttt{|...|})
    \item Mid-sentence unless forced by size constraints
\end{itemize}

\textbf{Alternative Strategies:}

\begin{itemize}
    \item \textbf{Semantic Chunking:} Embeds sentences, splits at $<0.7$ cosine similarity drops (100-200ms/page, requires embedding model)
    \item \textbf{Hierarchical Chunking:} Aligns chunks to sections/subsections from document hierarchy (20-40ms/page, requires clear structure)
\end{itemize}

\textbf{Chunk Enrichment:}

Each chunk receives metadata:
\begin{itemize}
    \item \textbf{Adjacent IDs:} Links to previous/next chunks for context expansion
    \item \textbf{Key Terms:} Pattern-matched capitalized phrases (top 10)
    \item \textbf{Summary:} First sentence or first 150 characters
    \item \textbf{Section Path:} Hierarchical path (Chapter $>$ Section $>$ Subsection)
    \item \textbf{Content Distribution:} Percentage breakdown by type (text, math, code, table)
\end{itemize}

\textbf{Validation:}

All chunks pass quality checks:
\begin{itemize}
    \item Size bounds (256-768 tokens)
    \item Complete sentences (no mid-sentence cuts)
    \item Balanced delimiters (matched brackets, quotes)
    \item Semantic coherence (not overly fragmented)
\end{itemize}

Invalid chunks are automatically corrected through merging (too small) or splitting at better boundaries (too large).

\subsubsection{Module 4: Embedding}

The embedding module generates dual representations: dense semantic vectors and sparse keyword vectors.

\textbf{Dense Embedding (BGE-small-en-v1.5):}

Neural embeddings capture semantic similarity:

\begin{lstlisting}[language=Python, caption={Dense embedding generation}]
from sentence_transformers import SentenceTransformer

model = SentenceTransformer('BAAI/bge-small-en-v1.5', device='cuda')

# Apply content-type prefixes
prefixed_text = f"{content_prefix}: {chunk_content}"

# Generate and normalize embeddings
embeddings = model.encode(
    prefixed_text,
    batch_size=64,  # GPU-optimized
    normalize_embeddings=True  # L2 normalization
)
\end{lstlisting}

\textbf{Batching Strategy:}
\begin{itemize}
    \item GPU: batch\_size=64 (optimal throughput)
    \item CPU: batch\_size=16 (memory-constrained)
    \item Dynamic adjustment on OOM errors
    \item Length-based sorting for efficient padding
\end{itemize}

\textbf{Sparse Embedding (BM25):}

Statistical keyword matching using Okapi BM25 \cite{robertson1995okapi}:

\begin{equation}
\text{score}(D,Q) = \sum_{i=1}^{n} \text{IDF}(q_i) \cdot \frac{f(q_i,D) \cdot (k_1+1)}{f(q_i,D) + k_1 \cdot (1-b+b \cdot \frac{|D|}{\text{avgdl}})}
\end{equation}

where $f(q_i,D)$ is term frequency, $|D|$ is document length, avgdl is average document length, $k_1=1.5$ (term frequency saturation), and $b=0.75$ (length normalization).

\textbf{Tokenization:}
\begin{itemize}
    \item Lowercase conversion
    \item Regex-based word splitting
    \item Stopword removal (English stopwords)
    \item Minimum token length: 3 characters
\end{itemize}

\textbf{Two-Tier Caching:}

To avoid redundant computation:

\begin{itemize}
    \item \textbf{L1 Memory Cache:} LRU cache (10,000 embeddings, <1ms access)
    \item \textbf{L2 Disk Cache:} SQLite database (unlimited capacity, 5-10ms access)
    \item \textbf{Cache Key:} SHA-256(model\_name + content)
    \item \textbf{Hit Rate:} 60-80\% for similar documents
    \item \textbf{Cleanup:} Automatic removal of entries >30 days old
\end{itemize}

\textbf{Fallback Mechanism:}

If BGE model fails:
\begin{enumerate}
    \item Attempt CPU fallback
    \item If still fails, use TF-IDF vectorization (384 dimensions via feature selection)
    \item Lower quality but ensures system never fails
\end{enumerate}

\subsubsection{Module 5: Indexing}

The indexing module builds multi-store architecture for fast, accurate retrieval.

\textbf{Vector Store (FAISS IndexHNSWFlat):}

Hierarchical Navigable Small World (HNSW) index \cite{malkov2018hnsw} for approximate nearest neighbor search:

\begin{itemize}
    \item \textbf{Index Type:} IndexHNSWFlat (no quantization, maximum accuracy)
    \item \textbf{M:} 32 connections per layer (memory vs. accuracy trade-off)
    \item \textbf{efConstruction:} 40 (build-time accuracy)
    \item \textbf{efSearch:} 16 (query-time accuracy, tunable)
    \item \textbf{Distance Metric:} Cosine similarity (via dot product on normalized vectors)
    \item \textbf{Build Time:} ~30 seconds for 100k vectors
    \item \textbf{Query Time:} <50ms for k=10 in 100k vectors
    \item \textbf{Memory:} ~2KB per vector (384-dim)
\end{itemize}

\textbf{Keyword Store (BM25):}

Pure statistical ranking:
\begin{itemize}
    \item \textbf{Implementation:} rank-bm25 library
    \item \textbf{Index Build:} 1 second per 10k documents
    \item \textbf{Query Time:} <10ms regardless of corpus size
    \item \textbf{Memory:} ~2KB per document
    \item \textbf{Score Normalization:} Divided by max score for [0,1] range
\end{itemize}

\textbf{Metadata Store (SQLite + FTS5):}

Relational database with full-text search:

\begin{lstlisting}[language=SQL, caption={Metadata store schema}]
CREATE TABLE chunks (
    id INTEGER PRIMARY KEY,
    doc_id TEXT NOT NULL,
    content TEXT NOT NULL,
    content_type TEXT,  -- text/math/code/table
    page_num INTEGER,
    section_path TEXT,
    has_math BOOLEAN,
    has_code BOOLEAN,
    token_count INTEGER,
    metadata TEXT  -- JSON blob
);

CREATE VIRTUAL TABLE chunks_fts USING fts5(
    content,
    content=chunks,
    content_rowid=id
);

CREATE INDEX idx_chunks_type ON chunks(content_type);
\end{lstlisting}

\textbf{Graph Store (NetworkX):}

Relationship graph for context expansion:
\begin{itemize}
    \item \textbf{Nodes:} Chunks with metadata
    \item \textbf{Edges:} Directed relationships (adjacent, explains, references) with weights
    \item \textbf{Traversal:} BFS for 2-3 hop expansion
    \item \textbf{Storage:} Pickle serialization (~50MB for 100k chunks)
\end{itemize}

\textbf{Index Optimization:}

Automatic tuning for production deployment:
\begin{itemize}
    \item \textbf{FAISS efSearch Tuning:} Target 95\% recall with minimal latency
    \item \textbf{SQLite Optimization:} ANALYZE, VACUUM, PRAGMA optimize
    \item \textbf{Validation:} Test queries measure latency improvements
\end{itemize}

\subsubsection{Module 6: Retrieval}

The retrieval module implements hybrid search combining semantic and lexical matching.

\textbf{Hybrid Retrieval Strategy:}

Reciprocal Rank Fusion (RRF) \cite{cormack2009rrf} combines dense and sparse results:

\begin{equation}
\text{RRF\_score}(d) = \sum_{r \in \{dense, sparse\}} \frac{1}{k + rank_r(d)}
\end{equation}

where $k=60$ (default RRF constant) and $rank_r(d)$ is the rank of document $d$ in ranking $r$.

\textbf{Retrieval Pipeline:}

\begin{enumerate}
    \item \textbf{Dense Search:} Embed query, retrieve top-20 from FAISS (50ms)
    \item \textbf{Sparse Search:} Tokenize query, retrieve top-20 from BM25 (10ms)
    \item \textbf{RRF Fusion:} Combine rankings with RRF scoring (10ms)
    \item \textbf{Filtering:} Apply score threshold (min\_score=0.3), diversity limits (max 2/section)
    \item \textbf{Graph Expansion:} Add adjacent chunks if similarity $>0.7$ (optional, 50ms)
    \item \textbf{Return:} Top-k final results (default k=5)
\end{enumerate}

\textbf{Query Analysis:}

Heuristic-based query understanding without ML:
\begin{itemize}
    \item \textbf{Content Type:} Math/code/table indicators for type-specific retrieval
    \item \textbf{Intent:} Definition/procedure/example/comparison patterns
    \item \textbf{Complexity:} Simple ($<10$ words), moderate (10-20), complex ($>20$)
\end{itemize}

\textbf{Two-Level Caching:}

\begin{itemize}
    \item \textbf{L1:} In-memory LRU (100 queries, <1ms)
    \item \textbf{L2:} SQLite persistent (10k+ queries, <10ms)
    \item \textbf{Key:} hash(query + k + filters)
    \item \textbf{TTL:} 3600 seconds (1 hour)
    \item \textbf{Hit Rate:} 30-40\% typical usage
\end{itemize}

\textbf{Performance Characteristics:}

\begin{itemize}
    \item \textbf{Hybrid:} 120-270ms (embedding + search + fusion)
    \item \textbf{Semantic only:} 100ms
    \item \textbf{Keyword only:} 11ms
    \item \textbf{Recall@5:} 85\% (hybrid), 70\% (semantic), 65\% (keyword)
\end{itemize}

\subsubsection{Module 7: Generation}

The generation module synthesizes answers from retrieved context using quantized Mistral-7B.

\textbf{Prompt Construction:}

Type-specific prompts optimized for Mistral-7B-Instruct format:

\begin{lstlisting}[caption={Mistral prompt template}]
<s>[INST] {system_prompt}

Context:
{retrieved_chunks_with_citations}

Question: {user_query}

{type_specific_instructions}
[/INST]
\end{lstlisting}

\textbf{System Prompts by Type:}

\begin{itemize}
    \item \textbf{General:} "You are a helpful educational assistant. Answer based ONLY on provided context. Cite sources using [1], [2] format."
    \item \textbf{Math:} "You are a mathematics tutor. Show step-by-step work. Explain each step clearly. Use proper notation. Cite source of formulas."
    \item \textbf{Code:} "You are a programming expert. Explain before showing code. Use markdown code blocks. Highlight language-specific features."
    \item \textbf{Table:} "You are a data analyst. Present findings clearly. Analyze patterns. Use structured format. Cite specific table rows."
\end{itemize}

\textbf{Context Building:}

Token budget management for 4k context window:
\begin{itemize}
    \item Total tokens: 4000
    \item System/prompt: ~200
    \item Context: ~2500 (retrieved chunks)
    \item Query: ~100
    \item Generation space: ~1200
\end{itemize}

\textbf{Chunk Selection:}
\begin{enumerate}
    \item Prioritize by relevance score (highest first)
    \item Prioritize by content type if query-specific
    \item Truncate chunks exceeding max\_chunk\_tokens (512)
    \item Select until budget exhausted
\end{enumerate}

\textbf{Answer Validation:}

Five-dimensional quality assessment:

\begin{enumerate}
    \item \textbf{Groundedness (50-70\%):} Word overlap between answer and context, citation patterns
    \item \textbf{Relevance (20-30\%):} Query term coverage, intent matching, non-answer detection
    \item \textbf{Completeness (10-20\%):} Length vs. query complexity (optimal: 500-1000 words)
    \item \textbf{Quality (10-15\%):} Capitalization, sentence structure, vocabulary diversity
    \item \textbf{Citations (5-10\%):} Presence of [1], [2] markers, sufficient source references (min 1, ideal 3+)
\end{enumerate}

Validation threshold: score $\geq 0.6$ and issues $\leq 2$ for acceptance.

\textbf{Post-Processing:}

\begin{itemize}
    \item \textbf{Text Cleaning:} Remove generation artifacts ([INST], </s>), normalize whitespace
    \item \textbf{Equation Formatting:} Normalize LaTeX markers ($\backslash$[ $\backslash$] $\rightarrow$ \$\$ \$\$)
    \item \textbf{Code Formatting:} Infer language, add syntax highlighting
    \item \textbf{Citation Extraction:} Map [1], [2] to chunk IDs and source metadata
\end{itemize}

\textbf{Fallback Strategy:}

If generation fails or produces low-quality output:
\begin{enumerate}
    \item \textbf{Extractive Summary:} Concatenate top 3 chunks with citations
    \item \textbf{Simple Answer:} Return best chunk verbatim
    \item \textbf{Chunk List:} Format as numbered list for user browsing
\end{enumerate}

Fallback selection based on error type (timeout $\rightarrow$ extractive, OOM $\rightarrow$ simple).

\subsection{Multimodal Extension}

To enable visual document understanding, we integrate Florence-2 \cite{xiao2024florence2} vision encoder with our quantized LLM through a "Vision-as-Language" architecture that avoids expensive multimodal training.

\subsubsection{Vision Agent Architecture}

Instead of projecting image embeddings into LLM hidden space, we employ textual translation:

\begin{enumerate}
    \item \textbf{Visual Ingestion:} Convert document pages to images
    \item \textbf{Dense Captioning:} Florence-2 generates detailed textual descriptions using compound prompting:
    \begin{itemize}
        \item \texttt{<OD>} (Object Detection): Identify labeled diagram components
        \item \texttt{<MORE\_DETAILED\_CAPTION>}: Generate paragraph-level semantic descriptions
    \end{itemize}
    \item \textbf{Context Fusion:} Structure descriptions as "[Image Context]: The figure shows... [Vision Agent Output]"
    \item \textbf{Cross-Modal Reasoning:} Quantized LLM processes combined visual + textual context
\end{enumerate}

\subsubsection{Vision Model Selection}

We evaluated six lightweight vision encoders on 50 tri-modal QA pairs (Image, Text, Question) stratified by complexity:

\begin{itemize}
    \item \textbf{Text-Only (Type A):} Control questions ensuring vision doesn't degrade text performance
    \item \textbf{Vision-Only (Type B):} Questions requiring visual interpretation (e.g., "What color represents...")
    \item \textbf{Combined (Type C):} Synthesis questions requiring both modalities
\end{itemize}

\textbf{Models Evaluated:}
\begin{itemize}
    \item Microsoft Florence-2 (Base \& Large): Unified model trained on FLD-5B
    \item Moondream2 (1.86B): Edge-optimized VLM
    \item BLIP-Base: Baseline image captioning
    \item GIT-Base: Generative Image-to-Text transformer
    \item ViT-GPT2: Classic encoder-decoder baseline
\end{itemize}


\newpage
\section{Results}

This section presents empirical results across three evaluation dimensions: computational efficiency (Section \ref{sec:efficiency}), task performance quality (Section \ref{sec:quality}), and RAG-specific capabilities (Section \ref{sec:rag}). All measurements are conducted on NVIDIA Tesla T4 (16GB VRAM) using Mistral-7B-v0.1 as the base model with standard quantization configurations (no calibration-aware optimization).

\subsection{Computational Efficiency Analysis}
\label{sec:efficiency}

\subsubsection{Latency Characterization}

Table \ref{tab:latency_detailed} presents comprehensive latency measurements across all quantization methods, decomposed into prefill (prompt processing) and decode (autoregressive generation) phases.

\begin{table}[H]
\centering
\caption{Latency measurements across quantization methods}
\label{tab:latency_detailed}
\small
\begin{tabular}{lccccc}
\toprule
\textbf{Method} & \textbf{TTFT} & \textbf{Decode} & \textbf{Prefill} & \textbf{Total} & \textbf{Speedup} \\
 & \textbf{(ms)} & \textbf{(ms/tok)} & \textbf{(ms)} & \textbf{(ms)} & \textbf{vs FP16} \\
\midrule
FP16 & & & & & 1.00$\times$ \\
NF4 & & & & & \\
GPTQ & & & & & \\
AWQ & & & & & \\
QuaRot & & & & & \\
HQQ & & & & & \\
KVQuant & & & & & \\
\bottomrule
\end{tabular}
\end{table}

\subsubsection{Throughput and Memory Efficiency}

Table \ref{tab:throughput_memory} characterizes sustained generation rates and memory consumption patterns.

\begin{table}[H]
\centering
\caption{Throughput and memory utilization}
\label{tab:throughput_memory}
\small
\begin{tabular}{lcccccc}
\toprule
\textbf{Method} & \textbf{Throughput} & \textbf{Model} & \textbf{Peak} & \textbf{Bits/} & \textbf{Memory} & \textbf{Compression} \\
 & \textbf{(tok/s)} & \textbf{Size (GB)} & \textbf{VRAM (MB)} & \textbf{Param} & \textbf{Efficiency} & \textbf{Ratio} \\
\midrule
FP16 & & 13.49 & & 16.0 & & 1.00$\times$ \\
NF4 & & & & 4.0 & & \\
GPTQ & & & & 4.0 & & \\
AWQ & & & & 4.0 & & \\
QuaRot & & & & 4.0 & & \\
HQQ & & & & 4.0 & & \\
KVQuant & & & & 16.0* & & \\
\bottomrule
\multicolumn{7}{l}{\footnotesize *KVQuant quantizes KV cache, not weights}
\end{tabular}
\end{table}

\subsubsection{Computational Efficiency Metrics}

Table \ref{tab:compute_efficiency} presents Model FLOPs Utilization and energy consumption estimates, revealing how effectively each method exploits available hardware capacity.

\begin{table}[H]
\centering
\caption{Computational efficiency and energy consumption}
\label{tab:compute_efficiency}
\small
\begin{tabular}{lcccc}
\toprule
\textbf{Method} & \textbf{FLOPs/tok} & \textbf{MFU} & \textbf{Energy} & \textbf{Energy} \\
 & \textbf{(GFLOPs)} & \textbf{(\%)} & \textbf{(mJ/tok)} & \textbf{Efficiency} \\
\midrule
FP16 & & & & 1.00$\times$ \\
NF4 & & & & \\
GPTQ & & & & \\
AWQ & & & & \\
QuaRot & & & & \\
HQQ & & & & \\
KVQuant & & & & \\
\bottomrule
\end{tabular}
\end{table}

\subsubsection{KV Cache Memory Analysis}

For methods with explicit KV cache management, Table \ref{tab:kv_cache_analysis} estimates cache sizes at different sequence lengths, demonstrating memory scaling behavior.

\begin{table}[H]
\centering
\caption{KV cache memory requirements by sequence length}
\label{tab:kv_cache_analysis}
\small
\begin{tabular}{lcccc}
\toprule
\textbf{Method} & \textbf{1K tokens} & \textbf{2K tokens} & \textbf{4K tokens} & \textbf{8K tokens} \\
 & \textbf{(MB)} & \textbf{(MB)} & \textbf{(MB)} & \textbf{(MB)} \\
\midrule
FP16 & & & & \\
NF4 & & & & \\
GPTQ & & & & \\
AWQ & & & & \\
QuaRot & & & & \\
HQQ & & & & \\
KVQuant & & & & \\
\bottomrule
\end{tabular}
\end{table}

\subsubsection{Batch Throughput Scaling}

Table \ref{tab:batch_throughput} examines throughput scaling across batch sizes, critical for understanding deployment capacity in production scenarios.

\begin{table}[H]
\centering
\caption{Batch throughput scaling (tokens/second)}
\label{tab:batch_throughput}
\small
\begin{tabular}{lcccc}
\toprule
\textbf{Method} & \textbf{Batch=1} & \textbf{Batch=2} & \textbf{Batch=4} & \textbf{Batch=8} \\
\midrule
FP16 & & & & \\
NF4 & & & & \\
GPTQ & & & & \\
AWQ & & & & \\
QuaRot & & & & \\
HQQ & & & & \\
KVQuant & & & & \\
\bottomrule
\end{tabular}
\end{table}

\subsubsection{Efficiency Summary and Pareto Analysis}

Table \ref{tab:efficiency_summary} consolidates key efficiency metrics for direct comparison.

\begin{table*}[H]
\centering
\caption{Consolidated efficiency metrics summary}
\label{tab:efficiency_summary}
\small
\begin{tabular}{lcccccccc}
\toprule
\textbf{Method} & \textbf{Latency} & \textbf{Throughput} & \textbf{TTFT} & \textbf{Memory} & \textbf{Model} & \textbf{MFU} & \textbf{Energy} & \textbf{Overall} \\
 & \textbf{(ms/tok)} & \textbf{(tok/s)} & \textbf{(ms)} & \textbf{(MB)} & \textbf{(GB)} & \textbf{(\%)} & \textbf{(mJ/tok)} & \textbf{Rank} \\
\midrule
FP16 & & & & & 13.49 & & & \\
NF4 & & & & & & & & \\
GPTQ & & & & & & & & \\
AWQ & & & & & & & & \\
QuaRot & & & & & & & & \\
HQQ & & & & & & & & \\
KVQuant & & & & & & & & \\
\bottomrule
\end{tabular}
\end{table*}

\subsection{Task Performance Analysis}
\label{sec:quality}

\subsubsection{Language Modeling: Perplexity}

Table \ref{tab:perplexity} presents perplexity measurements on WikiText-2, a foundational metric for assessing language modeling quality under compression.

\begin{table}[H]
\centering
\caption{Perplexity on WikiText-2 test set (100 samples)}
\label{tab:perplexity}
\small
\begin{tabular}{lcccc}
\toprule
\textbf{Method} & \textbf{Perplexity} & \textbf{Loss} & \textbf{Increase} & \textbf{Degradation} \\
 & & & \textbf{(abs)} & \textbf{(\%)} \\
\midrule
FP16 & 12.79 & & 0.00 & 0.00\% \\
NF4 & & & & \\
GPTQ & & & & \\
AWQ & & & & \\
QuaRot & & & & \\
HQQ & & & & \\
KVQuant & & & & \\
\bottomrule
\end{tabular}
\end{table}

\subsubsection{Commonsense Reasoning}

Table \ref{tab:commonsense} evaluates performance on commonsense reasoning benchmarks requiring world knowledge and intuitive understanding.

\begin{table}[H]
\centering
\caption{Commonsense reasoning performance (0-shot)}
\label{tab:commonsense}
\small
\begin{tabular}{lccccc}
\toprule
\textbf{Method} & \textbf{HellaSwag} & \textbf{ARC-Easy} & \textbf{ARC-Challenge} & \textbf{Average} & \textbf{Degradation} \\
 & \textbf{(acc\_norm)} & \textbf{(acc\_norm)} & \textbf{(acc\_norm)} & & \textbf{vs FP16} \\
\midrule
FP16 & 0.720 & 0.760 & 0.580 & 0.687 & 0.00\% \\
NF4 & & & & & \\
GPTQ & & & & & \\
AWQ & & & & & \\
QuaRot & & & & & \\
HQQ & & & & & \\
KVQuant & & & & & \\
\bottomrule
\end{tabular}
\end{table}

\subsubsection{Mathematical Reasoning}

Table \ref{tab:math_reasoning} assesses preservation of multi-step mathematical reasoning capabilities, often the most sensitive to quantization.

\begin{table}[H]
\centering
\caption{Mathematical reasoning performance on GSM8K (8-shot)}
\label{tab:math_reasoning}
\small
\begin{tabular}{lccc}
\toprule
\textbf{Method} & \textbf{Accuracy} & \textbf{Absolute} & \textbf{Relative} \\
 & \textbf{(exact match)} & \textbf{Drop} & \textbf{Degradation} \\
\midrule
FP16 & 0.360 & 0.000 & 0.0\% \\
NF4 & & & \\
GPTQ & & & \\
AWQ & & & \\
QuaRot & & & \\
HQQ & & & \\
KVQuant & & & \\
\bottomrule
\end{tabular}
\end{table}

\subsubsection{World Knowledge}

Table \ref{tab:world_knowledge} evaluates factual knowledge retention across 57 academic domains via MMLU.

\begin{table}[H]
\centering
\caption{World knowledge evaluation on MMLU (5-shot)}
\label{tab:world_knowledge}
\small
\begin{tabular}{lccc}
\toprule
\textbf{Method} & \textbf{Accuracy} & \textbf{Absolute} & \textbf{Relative} \\
 & & \textbf{Drop} & \textbf{Degradation} \\
\midrule
FP16 & 1.000 & 0.000 & 0.0\% \\
NF4 & & & \\
GPTQ & & & \\
AWQ & & & \\
QuaRot & & & \\
HQQ & & & \\
KVQuant & & & \\
\bottomrule
\end{tabular}
\end{table}

\subsubsection{Code Generation}

Table \ref{tab:code_generation} measures code synthesis quality via functional correctness on HumanEval.

\begin{table}[H]
\centering
\caption{Code generation performance on HumanEval (0-shot)}
\label{tab:code_generation}
\small
\begin{tabular}{lccc}
\toprule
\textbf{Method} & \textbf{Pass@1} & \textbf{Absolute} & \textbf{Relative} \\
 & & \textbf{Change} & \textbf{Change} \\
\midrule
FP16 & 0.050 & 0.000 & 0.0\% \\
NF4 & & & \\
GPTQ & & & \\
AWQ & & & \\
QuaRot & & & \\
HQQ & & & \\
KVQuant & & & \\
\bottomrule
\end{tabular}
\end{table}

\subsubsection{Cross-Task Performance Summary}

Table \ref{tab:task_summary} provides consolidated task performance with overall rankings.

\begin{table*}[H]
\centering
\caption{Comprehensive task performance summary}
\label{tab:task_summary}
\footnotesize
\begin{tabular}{lcccccccc}
\toprule
\textbf{Method} & \textbf{Perplexity} & \textbf{Commonsense} & \textbf{Math} & \textbf{Knowledge} & \textbf{Code} & \textbf{Average} & \textbf{Avg Deg.} & \textbf{Quality} \\
 & \textit{(↓)} & \textbf{Avg} & \textbf{(GSM8K)} & \textbf{(MMLU)} & \textbf{(HumanEval)} & \textbf{Accuracy} & \textbf{(\%)} & \textbf{Rank} \\
\midrule
FP16 & 12.79 & 0.687 & 0.360 & 1.000 & 0.050 & 0.524 & 0.0\% & 1 \\
NF4 & & & & & & & & \\
GPTQ & & & & & & & & \\
AWQ & & & & & & & & \\
QuaRot & & & & & & & & \\
HQQ & & & & & & & & \\
KVQuant & & & & & & & & \\
\bottomrule
\end{tabular}
\end{table*}

\subsubsection{Task-Specific Sensitivity Analysis}

Table \ref{tab:sensitivity_analysis} identifies which task categories exhibit highest and lowest sensitivity to quantization.

\begin{table}[H]
\centering
\caption{Task category sensitivity to quantization}
\label{tab:sensitivity_analysis}
\small
\begin{tabular}{lccc}
\toprule
\textbf{Task Category} & \textbf{Mean Degradation} & \textbf{Std Dev} & \textbf{Sensitivity} \\
 & \textbf{(across methods)} & & \textbf{Level} \\
\midrule
Mathematical Reasoning & & & \\
Code Generation & & & \\
Commonsense Reasoning & & & \\
World Knowledge & & & \\
Language Modeling & & & \\
\bottomrule
\end{tabular}
\end{table}

\subsection{RAG Performance Analysis}
\label{sec:rag}

\subsubsection{Answer Generation Quality}

Table \ref{tab:rag_answer_quality} evaluates generated answer quality using multiple complementary metrics.

\begin{table}[H]
\centering
\caption{RAG answer quality metrics}
\label{tab:rag_answer_quality}
\small
\begin{tabular}{lcccccc}
\toprule
\textbf{Method} & \textbf{F1} & \textbf{EM} & \textbf{ROUGE-1} & \textbf{ROUGE-L} & \textbf{BERTScore} & \textbf{Avg} \\
 & & & & & \textbf{F1} & \textbf{Score} \\
\midrule
FP16 & & & & & & \\
NF4 & & & & & & \\
GPTQ & & & & & & \\
AWQ & & & & & & \\
QuaRot & & & & & & \\
HQQ & & & & & & \\
KVQuant & & & & & & \\
\bottomrule
\end{tabular}
\end{table}

\subsubsection{Faithfulness and Relevance}

Table \ref{tab:rag_faithfulness} measures answer adherence to retrieved context and query relevance.

\begin{table}[H]
\centering
\caption{RAG faithfulness and relevance metrics}
\label{tab:rag_faithfulness}
\small
\begin{tabular}{lcccc}
\toprule
\textbf{Method} & \textbf{Faithfulness} & \textbf{Relevance} & \textbf{Avg Answer} & \textbf{Hallucination} \\
 & & & \textbf{Length} & \textbf{Rate} \\
\midrule
FP16 & & & & \\
NF4 & & & & \\
GPTQ & & & & \\
AWQ & & & & \\
QuaRot & & & & \\
HQQ & & & & \\
KVQuant & & & & \\
\bottomrule
\end{tabular}
\end{table}

\subsubsection{Retrieval Quality Assessment}

Table \ref{tab:retrieval_quality} characterizes the quality of context retrieval independent of generation.

\begin{table}[H]
\centering
\caption{Context retrieval quality metrics}
\label{tab:retrieval_quality}
\small
\begin{tabular}{lccccc}
\toprule
\textbf{Method} & \textbf{Context} & \textbf{Context} & \textbf{Context} & \textbf{Avg Score} & \textbf{Avg Context} \\
 & \textbf{Sufficiency} & \textbf{Precision} & \textbf{Coverage} & & \textbf{Length (chars)} \\
\midrule
FP16 & & & & & \\
NF4 & & & & & \\
GPTQ & & & & & \\
AWQ & & & & & \\
QuaRot & & & & & \\
HQQ & & & & & \\
KVQuant & & & & & \\
\bottomrule
\end{tabular}
\end{table}

\subsubsection{RAG vs. No-RAG Comparison}

Table \ref{tab:rag_comparison} quantifies the improvement gained from retrieval augmentation versus direct generation.

\begin{table}[H]
\centering
\caption{RAG improvement over no-RAG baseline}
\label{tab:rag_comparison}
\small
\begin{tabular}{lccccc}
\toprule
\textbf{Method} & \textbf{No-RAG} & \textbf{RAG} & \textbf{F1} & \textbf{Relative} & \textbf{RAG} \\
 & \textbf{F1} & \textbf{F1} & \textbf{Improvement} & \textbf{Improvement} & \textbf{Benefit} \\
\midrule
FP16 & & & & & \\
NF4 & & & & & \\
GPTQ & & & & & \\
AWQ & & & & & \\
QuaRot & & & & & \\
HQQ & & & & & \\
KVQuant & & & & & \\
\bottomrule
\end{tabular}
\end{table}

\subsubsection{RAG Efficiency Metrics}

Table \ref{tab:rag_efficiency} measures computational overhead introduced by retrieval augmentation.

\begin{table}[H]
\centering
\caption{RAG computational efficiency}
\label{tab:rag_efficiency}
\small
\begin{tabular}{lcccccc}
\toprule
\textbf{Method} & \textbf{Retrieval} & \textbf{RAG Gen} & \textbf{No-RAG Gen} & \textbf{RAG} & \textbf{No-RAG} & \textbf{Gen} \\
 & \textbf{Time (ms)} & \textbf{Time (ms)} & \textbf{Time (ms)} & \textbf{Throughput} & \textbf{Throughput} & \textbf{Slowdown} \\
\midrule
FP16 & & & & & & \\
NF4 & & & & & & \\
GPTQ & & & & & & \\
AWQ & & & & & & \\
QuaRot & & & & & & \\
HQQ & & & & & & \\
KVQuant & & & & & & \\
\bottomrule
\end{tabular}
\end{table}

\subsubsection{RAG Performance Summary}

Table \ref{tab:rag_summary} consolidates RAG metrics into overall scores for direct comparison.

\begin{table*}[H]
\centering
\caption{Comprehensive RAG performance summary}
\label{tab:rag_summary}
\small
\begin{tabular}{lcccccccc}
\toprule
\textbf{Method} & \textbf{Answer} & \textbf{Faithful-} & \textbf{Retrieval} & \textbf{RAG} & \textbf{RAG} & \textbf{Overall} & \textbf{RAG Deg.} & \textbf{RAG} \\
 & \textbf{Quality} & \textbf{ness} & \textbf{Quality} & \textbf{Improvement} & \textbf{Efficiency} & \textbf{RAG Score} & \textbf{vs FP16} & \textbf{Rank} \\
\midrule
FP16 & & & & & & & 0.0\% & 1 \\
NF4 & & & & & & & & \\
GPTQ & & & & & & & & \\
AWQ & & & & & & & & \\
QuaRot & & & & & & & & \\
HQQ & & & & & & & & \\
KVQuant & & & & & & & & \\
\bottomrule
\end{tabular}
\end{table*}

\subsection{Comparative Analysis}

\subsubsection{Efficiency-Quality Pareto Frontier}

Table \ref{tab:pareto_analysis} identifies methods that achieve optimal trade-offs between compression efficiency and task performance.

\begin{table}[H]
\centering
\caption{Pareto optimality analysis: efficiency vs. quality}
\label{tab:pareto_analysis}
\small
\begin{tabular}{lcccccc}
\toprule
\textbf{Method} & \textbf{Throughput} & \textbf{Memory} & \textbf{Task} & \textbf{RAG} & \textbf{Pareto} & \textbf{Use} \\
 & \textbf{Rank} & \textbf{Rank} & \textbf{Quality} & \textbf{Quality} & \textbf{Optimal} & \textbf{Case} \\
\midrule
FP16 & & & Baseline & Baseline & & \\
NF4 & & & & & & \\
GPTQ & & & & & & \\
AWQ & & & & & & \\
QuaRot & & & & & & \\
HQQ & & & & & & \\
KVQuant & & & & & & \\
\bottomrule
\end{tabular}
\end{table}

\subsubsection{Method Clustering Analysis}

Table \ref{tab:method_clustering} groups methods by behavioral similarity across all evaluation dimensions.

\begin{table}[H]
\centering
\caption{Method clustering by performance characteristics}
\label{tab:method_clustering}
\small
\begin{tabular}{llp{7cm}}
\toprule
\textbf{Cluster} & \textbf{Methods} & \textbf{Shared Characteristics} \\
\midrule
High Efficiency & & \\
Balanced & & \\
Quality-Focused & & \\
Specialized & & \\
\bottomrule
\end{tabular}
\end{table}

\subsubsection{Hardware Compatibility Summary}

Table \ref{tab:hardware_compatibility} documents observed hardware-specific behaviors on Tesla T4 (Turing architecture).

\begin{table}[H]
\centering
\caption{Hardware compatibility assessment for Tesla T4}
\label{tab:hardware_compatibility}
\small
\begin{tabular}{lccccc}
\toprule
\textbf{Method} & \textbf{Native} & \textbf{Kernel} & \textbf{Observed} & \textbf{T4} & \textbf{Deployment} \\
 & \textbf{Support} & \textbf{Fallback} & \textbf{Issues} & \textbf{Viable} & \textbf{Notes} \\
\midrule
FP16 & & & & & \\
NF4 & & & & & \\
GPTQ & & & & & \\
AWQ & & & & & \\
QuaRot & & & & & \\
HQQ & & & & & \\
KVQuant & & & & & \\
\bottomrule
\end{tabular}
\end{table}

\subsubsection{Deployment Decision Framework}

Table \ref{tab:deployment_recommendations} provides actionable guidance for method selection based on deployment priorities.

\begin{table*}[H]
\centering
\caption{Method selection recommendations by deployment scenario}
\label{tab:deployment_recommendations}
\small
\begin{tabular}{lp{10cm}l}
\toprule
\textbf{Deployment Scenario} & \textbf{Recommended Method} & \textbf{Rationale} \\
\midrule
Maximum Quality Preservation & & \\
Minimum Memory Footprint & & \\
Maximum Throughput & & \\
Long-Context RAG (>4K tokens) & & \\
Hardware-Constrained (Turing GPUs) & & \\
Zero-Shot Deployment & & \\
Balanced Performance & & \\
\bottomrule
\end{tabular}
\end{table*}

\subsection{Key Findings Summary}

Based on the comprehensive evaluation results, we identify several critical insights:

\begin{table}[H]
\centering
\caption{Summary of key findings}
\label{tab:key_findings}
\small
\begin{tabular}{lp{10cm}}
\toprule
\textbf{Finding Category} & \textbf{Key Insight} \\
\midrule
Best Overall Method & \\
Lowest Degradation & \\
Highest Efficiency & \\
Best for RAG & \\
Best for Math & \\
Best for T4 Hardware & \\
Task Sensitivity Ranking & \\
Compression-Quality Sweet Spot & \\
\bottomrule
\end{tabular}
\end{table}

% \section{Discussion}

\subsection{Hardware-Algorithm Compatibility as Primary Determinant}

\subsubsection{Kernel Dependency Impact}

Our results reveal that hardware-algorithm compatibility dominates practical performance far more than theoretical compression ratios. [Discuss specific findings about GPTQ/HQQ fallback on T4 vs. NF4/AWQ stability]

\textbf{Key Finding:} [Fill in observation about throughput differences despite identical 4-bit compression]

\textbf{Hypothesis:} GPTQ and HQQ rely on INT4 tensor cores and optimized GEMM kernels introduced in Ampere architecture (compute capability 8.0+). The Tesla T4 (Turing, compute capability 7.5) lacks these specialized operations, causing fallback to inefficient FP16 emulation paths that negate quantization benefits.

\textbf{Evidence:} [Reference specific metrics from efficiency tables showing 10-18× slowdown]

\textbf{Implications for Practitioners:}
\begin{itemize}
\item For Turing-generation hardware (T4, RTX 20-series), distribution-based quantization (NF4) is not just preferable but the \textit{only viable option}
\item Kernel-dependent methods should be reserved for Ampere+ architectures where specialized INT4 operations are available
\item Hardware specification must be the first consideration when selecting quantization strategies, superseding theoretical compression ratios
\end{itemize}

\subsubsection{Architecture-Specific Optimization}

[Discuss why NF4 performs better on T4 - likely due to its reliance on standard FP16 operations rather than specialized kernels]

\subsection{Task-Dependent Quantization Sensitivity}

\subsubsection{Differential Impact Across Domains}

Our evaluation reveals striking heterogeneity in how quantization affects different cognitive capabilities:

\paragraph{Preserved Capabilities}
\begin{itemize}
\item \textbf{Knowledge Retrieval:} [Discuss ARC-Easy/Challenge results showing minimal degradation]
\item \textbf{Commonsense Reasoning:} [Analyze HellaSwag robustness]
\item \textbf{Code Generation:} [Explain HumanEval stability]
\end{itemize}

\paragraph{Degraded Capabilities}
\begin{itemize}
\item \textbf{Mathematical Reasoning:} [Discuss GSM8K showing 25\% drop - explain why multi-step arithmetic is vulnerable]
\end{itemize}

\textbf{Mechanistic Hypothesis:} Quantization appears to affect \textit{deep reasoning chains} more severely than \textit{pattern matching} or \textit{retrieval-based} tasks. Mathematical problem-solving requires maintaining precise intermediate states across multiple reasoning steps, while knowledge retrieval and commonsense reasoning rely more on learned associations that may be more robust to precision loss.

\subsubsection{Implications for Application Design}

\begin{itemize}
\item For knowledge-intensive RAG applications (document QA, fact retrieval), aggressive 4-bit quantization is viable with minimal quality loss
\item For reasoning-heavy applications (mathematical tutoring, logical problem-solving), either:
  \begin{itemize}
  \item Use higher precision (FP16 or conservative 4-bit with careful method selection)
  \item Employ hybrid precision strategies (discussed below)
  \item Implement post-quantization fine-tuning on reasoning tasks
  \end{itemize}
\item Code generation appears surprisingly robust, suggesting that programming tasks rely more on learned patterns than arithmetic precision
\end{itemize}

\subsection{The NF4 vs. AWQ Trade-off}

\subsubsection{Marginal Efficiency Gains}

AWQ demonstrates [X\%] better throughput and [Y\%] lower energy consumption compared to NF4. [Fill in specific numbers from results]

\textbf{However,} this efficiency advantage comes at the cost of:
\begin{itemize}
\item Higher perplexity degradation ([X\%] vs [Y\%])
\item Potentially lower task accuracy ([specific tasks where AWQ underperforms])
\item [Any RAG-specific degradation observed]
\end{itemize}

\subsubsection{When to Choose AWQ}

AWQ may be preferable in scenarios where:
\begin{enumerate}
\item \textbf{Latency is Critical:} Applications requiring sub-[X]ms response times where the [Y\%] speedup is meaningful
\item \textbf{Energy Budget is Constrained:} Battery-powered edge devices where [Z] mJ/token reduction significantly extends operation time
\item \textbf{Perplexity Degradation is Acceptable:} Tasks where the [X\%] perplexity increase does not materially impact user experience
\item \textbf{Simple Retrieval Tasks:} RAG applications focused on straightforward fact extraction rather than complex reasoning
\end{enumerate}

\subsubsection{General Recommendation: NF4}

For general-purpose deployment, NF4 provides superior balance:
\begin{itemize}
\item More stable performance across diverse tasks
\item Better perplexity preservation ([X\%] vs [Y\%])
\item Comparable RAG performance
\item Universal hardware compatibility
\item Only [Z\%] efficiency penalty
\end{itemize}

The [Z\%] efficiency cost is a worthwhile trade for [X\%] better quality preservation in most production scenarios.

\subsection{RAG-Specific Insights}

\subsubsection{Context Grounding Under Quantization}

[Analyze context adherence results]

\textbf{Key Finding:} [Discuss whether quantization increases hallucination rate]

\textbf{Surprising Result:} [If applicable, note if quantized models show better/worse context precision than expected]

\textbf{Practical Implication:} [Discuss whether quantized models are safe for RAG applications where faithfulness is critical]

\subsubsection{Numerical Precision Preservation}

[Analyze numerical accuracy results]

\textbf{Critical for:}
\begin{itemize}
\item Financial document QA
\item Medical report analysis
\item Scientific literature retrieval
\item Legal document review
\end{itemize}

\textbf{Finding:} [Discuss whether quantization corrupts exact numerical extraction]

\subsubsection{Multi-Hop Reasoning Degradation}

[Analyze multi-hop results]

\textbf{Hypothesis:} Complex reasoning requiring synthesis across multiple context pieces may be more vulnerable to quantization because:
\begin{enumerate}
\item Requires maintaining multiple entity references simultaneously
\item Involves chaining logical steps that accumulate precision errors
\item Demands stronger attention weights that may be affected by quantization
\end{enumerate}

\textbf{Mitigation Strategies:}
\begin{itemize}
\item [Suggest approaches based on results]
\end{itemize}

\subsubsection{Context Window Effects}

[Analyze position degradation, long context, and distractor results]

\textbf{Position Bias:} [Discuss if quantization exacerbates "lost in the middle" phenomenon]

\textbf{Length Scaling:} [Analyze if quantization affects long-context utilization]

\textbf{Noise Robustness:} [Discuss distractor resistance]

\subsubsection{Instruction Following}

[Analyze constraint adherence and format compliance]

\textbf{Finding:} [Discuss whether quantization affects ability to follow specific constraints]

\textbf{Implication:} [Discuss reliability for production RAG systems requiring structured outputs]

\subsection{Aggressive Quantization Analysis (2-bit and 3-bit)}

\subsubsection{GPTQ Bit-Width Progression}

[Analyze the 4-bit â†' 3-bit â†' 2-bit degradation curve]

\textbf{Compression vs. Quality Trade-off:}
\begin{itemize}
\item 4-bit: [X]× compression, [Y\%] quality loss
\item 3-bit: [X]× compression, [Y\%] quality loss
\item 2-bit: [X]× compression, [Y\%] quality loss
\end{itemize}

\textbf{Viability Assessment:}
\begin{itemize}
\item \textbf{2-bit:} [Evaluate if usable for any practical application]
\item \textbf{3-bit:} [Discuss sweet spot analysis - worth the extra compression over 4-bit?]
\end{itemize}

\subsubsection{Failure Modes at Extreme Quantization}

[Discuss specific failure patterns observed at 2-bit and 3-bit]

\textbf{Task Collapse:} [Which tasks become unusable first?]

\textbf{RAG Reliability:} [At what bit-width does RAG performance become unacceptable?]

\subsection{Perplexity as Quality Proxy}

\subsubsection{Correlation Analysis}

[Compare perplexity rankings with task performance rankings]

\textbf{Finding:} Perplexity shows [strong/weak/moderate] correlation with downstream task performance ([correlation coefficient if calculated])

\textbf{Discrepancies:} [Identify cases where low perplexity doesn't translate to task accuracy]

\subsubsection{Limitations of Perplexity}

Our results confirm that perplexity alone is insufficient for evaluating quantized models because:
\begin{itemize}
\item [Specific examples from results]
\item Does not capture task-specific degradation patterns
\item May not reflect RAG-critical capabilities like context grounding
\item Can mask differential sensitivity across domains
\end{itemize}

\textbf{Recommendation:} Always supplement perplexity with task-specific and application-specific evaluation.

\subsection{Deployment Decision Framework}

Based on our comprehensive evaluation, we propose the following decision framework:

\subsubsection{Hardware Compatibility First}

\begin{table}[ht]
\centering
\caption{Hardware-Based Method Selection}
\begin{adjustbox}{max width=\columnwidth}
\small
\begin{tabular}{lp{6cm}}
\toprule
\textbf{Hardware} & \textbf{Recommended Method} \\
\midrule
Turing (T4, RTX 20xx) & NF4 (only viable 4-bit option) \\
Ampere (A100, RTX 30xx) & Test AWQ and GPTQ; choose based on workload \\
Ada (RTX 40xx) & AWQ or GPTQ for maximum efficiency \\
Hopper (H100) & All methods viable; optimize for workload \\
\bottomrule
\end{tabular}
\end{adjustbox}
\end{table}

\subsubsection{Application-Specific Guidance}

\begin{table}[ht]
\centering
\caption{Application-Based Method Selection}
\begin{adjustbox}{max width=\columnwidth}
\small
\begin{tabular}{lp{6cm}}
\toprule
\textbf{Application Type} & \textbf{Recommendation} \\
\midrule
Document QA (simple) & Any 4-bit method; prioritize efficiency \\
Mathematical tutoring & NF4 + fine-tuning or FP16 for critical steps \\
Code generation & Any 4-bit method; appears robust \\
Multi-hop reasoning & NF4 with higher quality preservation \\
Numerical extraction & Test carefully; consider hybrid precision \\
Long-context retrieval & Any 4-bit; monitor position bias \\
Structured output generation & Test instruction following; NF4 safest \\
\bottomrule
\end{tabular}
\end{adjustbox}
\end{table}

\subsubsection{Optimization Priority}

\begin{table}[ht]
\centering
\caption{Priority-Based Method Selection}
\begin{adjustbox}{max width=\columnwidth}
\small
\begin{tabular}{lp{6cm}}
\toprule
\textbf{Priority} & \textbf{Recommendation} \\
\midrule
Maximum throughput & AWQ (if hardware supports) \\
Maximum compression & GPTQ 2/3-bit (evaluate quality carefully) \\
Best quality preservation & NF4 \\
Energy efficiency & AWQ (marginal gains) \\
General-purpose deployment & NF4 (best balance) \\
Production reliability & NF4 (hardware-agnostic) \\
\bottomrule
\end{tabular}
\end{adjustbox}
\end{table}

\subsection{Hybrid Precision Strategies}

Given the differential task sensitivity observed, we propose hybrid precision architectures:

\subsubsection{Layer-Selective Quantization}

\begin{itemize}
\item \textbf{Preserve in FP16:} Early attention layers (input processing), final output layers
\item \textbf{Aggressive quantization (4-bit):} Middle feedforward layers (pattern matching)
\item \textbf{Rationale:} [Explain based on where degradation is most severe]
\end{itemize}

\subsubsection{Task-Adaptive Precision}

\begin{itemize}
\item \textbf{Route to FP16:} Mathematical reasoning queries
\item \textbf{Route to 4-bit:} Factual retrieval, commonsense reasoning
\item \textbf{Implementation:} Lightweight classifier to detect query type
\end{itemize}

\subsubsection{Dynamic Precision Scaling}

\begin{itemize}
\item Start with 4-bit for efficiency
\item If uncertainty exceeds threshold, re-run critical steps in FP16
\item Balances efficiency with quality assurance
\end{itemize}

\subsection{Limitations and Future Work}

\subsubsection{Study Limitations}

\begin{enumerate}
\item \textbf{Single Model:} Evaluation focused on Mistral-7B; patterns may differ for other architectures
\item \textbf{Single Hardware:} Tesla T4 results may not generalize to all edge devices
\item \textbf{Synthetic RAG Tasks:} Some RAG evaluations use constructed datasets rather than real-world data
\item \textbf{Limited Bit Widths:} Did not explore mixed-precision or sub-2-bit quantization
\item \textbf{No Fine-Tuning:} Evaluated only zero-shot quantization without post-quantization adaptation
\end{enumerate}

\subsubsection{Future Research Directions}

\paragraph{Expanded Model Coverage}
\begin{itemize}
\item Evaluate Llama 3, Phi, Gemma model families
\item Compare 7B vs. 13B vs. 70B parameter scaling
\item Assess specialized models (code, math, multilingual)
\end{itemize}

\paragraph{Hardware Diversity}
\begin{itemize}
\item Mobile processors (Snapdragon, Apple Silicon)
\item AMD GPUs (ROCm compatibility)
\item Intel GPUs (Arc series)
\item NPUs and ASICs (Coral, Hailo)
\end{itemize}

\paragraph{Advanced Quantization Techniques}
\begin{itemize}
\item Quantization-aware fine-tuning (QAT)
\item Mixed-precision strategies
\item Dynamic quantization
\item Outlier-aware methods (SpQR, OWQ)
\item Sub-2-bit extreme quantization
\end{itemize}

\paragraph{RAG System Integration}
\begin{itemize}
\item End-to-end RAG pipeline evaluation (retriever + LLM)
\item Real-world document corpus testing
\item Multi-turn conversation evaluation
\item Embedding model quantization impact
\item Vector database integration overhead
\end{itemize}

\paragraph{Production Deployment Studies}
\begin{itemize}
\item Long-running stability testing
\item Multi-user concurrent request handling
\item Model switching overhead (FP16 â†" 4-bit)
\item Thermal throttling under sustained load
\item Battery life impact on mobile devices
\end{itemize}

\paragraph{Theoretical Understanding}
\begin{itemize}
\item Why does mathematical reasoning degrade more?
\item What causes hardware-specific fallback behaviors?
\item Can we predict task sensitivity from model architecture?
\item Optimal bit allocation across layers
\end{itemize}

\subsection{Broader Implications}

\subsubsection{Democratization of LLMs}

Effective quantization enables:
\begin{itemize}
\item Deployment on consumer hardware (<\$500 GPUs)
\item Local inference without cloud dependencies
\item Privacy-preserving on-device processing
\item Reduced operational costs for startups
\end{itemize}

\subsubsection{Environmental Impact}

[Calculate and discuss potential carbon footprint reduction from efficient quantization deployment]

\subsubsection{Research Methodology}

This study demonstrates the necessity of:
\begin{itemize}
\item Hardware-aware evaluation
\item Task-specific metrics beyond perplexity
\item Application-focused benchmarking (RAG)
\item Comprehensive efficiency profiling
\end{itemize}

These principles should guide future model compression research.

% \section{Conclusion}

This study provides the first comprehensive hardware-aware benchmarking of modern quantization methods on consumer-grade edge hardware. Our key contributions include:

\begin{enumerate}
\item \textbf{Hardware Compatibility Analysis:} Demonstrating that algorithm-hardware compatibility is the primary determinant of practical performance, with kernel-dependent methods experiencing 10-18× slowdown on Turing GPUs.

\item \textbf{Task-Specific Evaluation:} Establishing that quantization impacts tasks differentially, with mathematical reasoning showing 25\% degradation while knowledge retrieval remains robust.

\item \textbf{Prescriptive Guidelines:} Identifying NF4 as the optimal method for Tesla T4 and similar hardware, achieving 3.6× compression with minimal performance loss and superior hardware compatibility.
\end{enumerate}

For Turing-generation hardware, NF4 represents the \textbf{only viable quantization method}, achieving stable 12.30 tok/s throughput with reasonable energy consumption (3975 mJ/token). While AWQ offers marginal efficiency gains, NF4's superior task performance and hardware stability make it the recommended choice for general-purpose edge deployment.

\textbf{Future Work:} Investigating hybrid precision strategies where critical layers (attention heads for reasoning tasks) retain higher precision while feedforward networks are aggressively quantized may resolve the mathematical reasoning degradation observed in our study.


% \section*{Acknowledgments}

This research was conducted at Egypt-Japan University of Science and Technology (E-JUST). We thank the Anthropic team for access to computational resources.


% \bibliographystyle{IEEEtran}
% \bibliography{references}

\end{document}