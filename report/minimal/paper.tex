\documentclass[11pt,a4paper,onecolumn]{article}

\usepackage[margin=1in]{geometry}
\usepackage[T1]{fontenc}
\usepackage[utf8]{inputenc}
\usepackage{times}
\usepackage{setspace}
\usepackage{url}
\usepackage{graphicx}
\usepackage{amsmath, amssymb, amsthm}
\usepackage{booktabs, multirow, makecell, threeparttable}
\usepackage{algorithm, algorithmic}
\usepackage{float}
\usepackage{caption, subcaption}
\usepackage{xcolor}
\usepackage{tikz}
\usepackage{hyperref}
\usepackage{listings}
\usepackage{pdflscape}
\usepackage{longtable}
\usepackage{enumitem}
\usepackage[scaled=0.9]{beramono}

\usetikzlibrary{shapes.geometric, arrows.meta, positioning, calc, backgrounds, fit}

\hypersetup{
    colorlinks = true,
    linkcolor = black,
    urlcolor = blue,
    citecolor = black
}

\lstset{
  language=Python,
  basicstyle=\ttfamily\footnotesize,
  numbers=left,
  numberstyle=\tiny\color{gray!70},
  stepnumber=1,
  numbersep=8pt,
  frame=lines,
  rulecolor=\color{black!40},
  backgroundcolor=\color{white},
  keywordstyle=\color{blue},
  commentstyle=\color{green!50!black},
  stringstyle=\color{red!70!black},
  showstringspaces=false,
  tabsize=2,
  breaklines=true,
  captionpos=b,
  aboveskip=8pt,
  belowskip=8pt
}

\title{\Large\bfseries
Compression Techniques for Mistral-7B:\\
A Comprehensive Evaluation of Efficiency and Performance
}

\author{
  Zahraa Selim \quad
  Menna Hamed \quad
  Wesam Ahmed \quad
  Sohyla Said \quad
  Sara Basheer \\[1.5ex]
  \normalsize\textit{Under the supervision of} \\[.5ex]
  Prof. Rami Zewail \\[1.5ex]
  \normalsize Department of Computer Science and Engineering,\\
  Egypt-Japan University of Science and Technology (E-JUST)
}

\date{}
\setlength{\parindent}{0pt}

\begin{document}
\maketitle
\onehalfspacing

\section{Experimental Setup}

\subsection{Overview}
This study conducts a comprehensive comparative evaluation of multiple compression techniques applied to the Mistral-7B model. Our experimental protocol follows a two-phase approach: (1) compress the base model and evaluate compression impact, and (2) fine-tune the compressed models on downstream tasks and re-evaluate to measure performance recovery. All experiments are conducted under resource-constrained environments to reflect realistic deployment scenarios.

\subsection{Environment and Resources}
All experiments were conducted on cloud-based platforms with the following specifications:
\begin{itemize}
    \item \textbf{Platforms:} Kaggle and Google Colab free-tier instances
    \item \textbf{Hardware:} NVIDIA Tesla T4 GPU (16GB VRAM)
    \item \textbf{Software:} PyTorch 2.x, Transformers 4.x, bitsandbytes, AutoGPTQ, AutoAWQ
    \item \textbf{Base Model:} Mistral-7B (FP16 baseline, 13.49 GB model size)
\end{itemize}

These resource constraints (limited VRAM and compute) reflect the target deployment scenario for compressed models and ensure our findings are applicable to practical edge and consumer-grade hardware settings.

\newpage
\subsection{Compression Techniques}

\subsubsection{Quantization Methods}
We evaluate four state-of-the-art post-training quantization techniques:

\begin{itemize}
    \item \textbf{NF4 (4-bit NormalFloat):} Uses an information-theoretically optimal data type for normally distributed weights, implemented via bitsandbytes with double quantization enabled.
    
    \textbf{Parameters:}
    \begin{itemize}
        \item Quantization type: NF4 (NormalFloat4)
        \item Double quantization: Enabled
        \item Compute dtype: float16
        \item Group size: Per-tensor (default)
    \end{itemize}
    
    \textbf{Implementation:}
\begin{lstlisting}[language=Python, caption={NF4 Quantization with BitsAndBytes}]
from transformers import BitsAndBytesConfig, AutoModelForCausalLM
import torch

nf4_config = BitsAndBytesConfig(
    load_in_4bit=True,
    bnb_4bit_quant_type="nf4",
    bnb_4bit_use_double_quant=True,
    bnb_4bit_compute_dtype=torch.float16
)

model = AutoModelForCausalLM.from_pretrained(
    "mistralai/Mistral-7B-Instruct-v0.1",
    quantization_config=nf4_config,
    device_map="auto"
)
\end{lstlisting}

\vspace{0.5em}

    \item \textbf{GPTQ:} Layer-wise quantization that minimizes reconstruction error using Hessian information, applied at 4-bit precision with group size of 128. We use TheBloke's pre-quantized model to avoid the computationally expensive quantization process.
    
    \textbf{Parameters:}
    \begin{itemize}
        \item Bits: 4
        \item Group size: 128
        \item Dataset: C4 (used during quantization)
        \item Activation order: Optimized via Hessian
    \end{itemize}
    
    \textbf{Implementation:}
\begin{lstlisting}[language=Python, caption={GPTQ Quantization (Pre-quantized Model)}]
from auto_gptq import AutoGPTQForCausalLM

model = AutoGPTQForCausalLM.from_quantized(
    "TheBloke/Mistral-7B-Instruct-v0.1-GPTQ",
    device="cuda:0",
    use_safetensors=True
)
\end{lstlisting}

\vspace{0.5em}

    \item \textbf{AWQ (Activation-aware Weight Quantization):} Protects salient weights based on activation magnitudes, using 4-bit quantization with per-channel scaling. Similar to GPTQ, we use a pre-quantized model.
    
    \textbf{Parameters:}
    \begin{itemize}
        \item Bits: 4
        \item Group size: 128
        \item Zero point: True
        \item Version: GEMM (optimized matrix multiplication)
    \end{itemize}
    
    \textbf{Implementation:}
\begin{lstlisting}[language=Python, caption={AWQ Quantization (Pre-quantized Model)}]
from awq import AutoAWQForCausalLM

model = AutoAWQForCausalLM.from_quantized(
    "TheBloke/Mistral-7B-Instruct-v0.1-AWQ",
    fuse_layers=True,
    safetensors=True
)
\end{lstlisting}

\vspace{0.5em}

    \item \textbf{HQQ (Half-Quadratic Quantization):} Fast quantization method optimizing a custom loss function, configured for 4-bit weights with optimized zero-point placement. This method quantizes the model on-the-fly during loading.
    
    \textbf{Parameters:}
    \begin{itemize}
        \item Bits: 4
        \item Group size: 64
        \item Axis: 1 (row-wise quantization)
        \item Compute dtype: float16
    \end{itemize}
    
    \textbf{Implementation:}
\begin{lstlisting}[language=Python, caption={HQQ Quantization (On-the-fly)}]
from hqq.models.hf.base import AutoHQQHFModel
from hqq.core.quantize import BaseQuantizeConfig

quant_config = BaseQuantizeConfig(
    nbits=4,
    group_size=64,
    axis=1
)

model = AutoHQQHFModel.from_pretrained(
    "mistralai/Mistral-7B-Instruct-v0.1",
    torch_dtype=torch.float16
)

model.quantize_model(
    quant_config=quant_config,
    device='cuda'
)
\end{lstlisting}

\end{itemize}

\vspace{1em}
\noindent All quantization methods target 4-bit precision to achieve similar compression ratios (approximately 3.6x) for fair comparison. NF4 and HQQ quantize on-the-fly during model loading, while GPTQ and AWQ use pre-quantized checkpoints from TheBloke's repository for efficiency.

\newpage
\subsection{Fine-tuning Strategy}

\subsubsection{Overview}
Following compression, we apply targeted fine-tuning to recover performance degradation and adapt models to downstream tasks. Our fine-tuning strategy focuses on six key capability domains, selected to comprehensively evaluate the impact of quantization across diverse reasoning patterns and knowledge types.

\subsubsection{Fine-tuning Domains and Datasets}

We organize fine-tuning and evaluation around six capability categories, each testing different aspects of model knowledge and reasoning under compression:

\paragraph{Mathematical Reasoning}
\textbf{Domains:} Mathematics, physics, formal logic \\
\textbf{What It Tests:} Symbolic manipulation and logical reasoning under quantization. This category evaluates whether compressed models retain the ability to perform multi-step arithmetic, algebraic manipulation, and formal reasoning chains. \\
\textbf{Datasets:}
\begin{itemize}
    \item \textbf{GSM8K:} Grade-school math word problems requiring multi-step reasoning (8-shot evaluation, exact match metric)
    \item \textbf{MATH:} Competition-level mathematics across algebra, geometry, number theory, and calculus (4-shot evaluation with majority voting)
\end{itemize}

\paragraph{Code Generation}
\textbf{Domains:} Python programming, algorithms, debugging \\
\textbf{What It Tests:} Syntax precision combined with algorithmic reasoning. Quantization can particularly impact code generation due to the need for exact token sequences and structured output. \\
\textbf{Datasets:}
\begin{itemize}
    \item \textbf{HumanEval:} 164 hand-written Python programming problems with unit test evaluation (0-shot, pass@1 metric)
    \item \textbf{MBPP (Mostly Basic Python Problems):} 974 entry-level Python tasks with train/test splits (3-shot, pass@1 metric)
    \item \textbf{CodeAlpaca:} 20,000 instruction-tuning examples for code generation tasks, used for fine-tuning
\end{itemize}

\paragraph{World Knowledge}
\textbf{Domains:} Science, history, geography, general knowledge \\
\textbf{What It Tests:} Factual recall and conceptual understanding. This category assesses whether quantization affects the model's ability to retain and retrieve stored knowledge about the world. \\
\textbf{Datasets:}
\begin{itemize}
    \item \textbf{MMLU (Massive Multitask Language Understanding):} 57 academic subjects spanning STEM, humanities, and social sciences (5-shot, accuracy metric)
    \item \textbf{TriviaQA:} Large-scale reading comprehension dataset with 95,000 question-answer pairs (5-shot, exact match metric)
\end{itemize}

\paragraph{Domain Expertise}
\textbf{Domains:} Medicine, law, finance, specialized technical fields \\
\textbf{What It Tests:} Specialized terminology and domain-specific reasoning patterns. Professional domains require both technical vocabulary preservation and complex reasoning chains. \\
\textbf{Datasets:}
\begin{itemize}
    \item \textbf{MedQA:} Medical exam questions in USMLE style, testing clinical reasoning and medical knowledge
    \item \textbf{LegalBench:} Suite of legal reasoning tasks including contract interpretation, precedent analysis, and statutory reasoning
    \item \textbf{ArXiv Custom:} Custom-built evaluation sets from scientific papers in the target domain
\end{itemize}

\paragraph{Language Understanding and Summarization}
\textbf{Domains:} Reading comprehension, text summarization \\
\textbf{What It Tests:} Coherence and compression ability. This evaluates whether compressed models can maintain fluency and capture key information when generating or condensing text. \\
\textbf{Datasets:}
\begin{itemize}
    \item \textbf{CNN/DailyMail:} News article summarization benchmark with 300,000+ article-summary pairs
    \item \textbf{BoolQ:} Yes/no question answering requiring reading comprehension (0-shot, accuracy metric)
    \item \textbf{QuAC:} Conversational question answering with context (0-shot, F1 metric)
\end{itemize}

\paragraph{Instruction Following}
\textbf{Domains:} Format control, role-play, task specification adherence \\
\textbf{What It Tests:} Model steerability post-quantization. Instruction following is critical for real-world deployment, where models must precisely follow user directives and formatting requirements. \\
\textbf{Datasets:}
\begin{itemize}
    \item \textbf{Alpaca:} 52,000 instruction-following examples covering diverse task types and formats
    \item \textbf{BBH (Big Bench Hard):} 23 challenging tasks from BigBench (3-shot, accuracy metric)
    \item \textbf{AGI Eval:} Human-level exam questions testing general intelligence (3-5 shot, accuracy metric)
\end{itemize}

\subsubsection{Fine-tuning Methodology}

\textbf{Training Configuration:}
We employ parameter-efficient fine-tuning techniques to avoid catastrophic forgetting and reduce computational requirements:

\begin{table}[H]
\centering
\caption{Fine-tuning hyperparameters}
\label{tab:finetuning_params}
\small
\begin{tabular}{lll}
\hline
\textbf{Parameter} & \textbf{Value} & \textbf{Description} \\
\hline
Method & LoRA & Low-Rank Adaptation \\
LoRA rank & 8 & Rank of adaptation matrices \\
LoRA alpha & 16 & Scaling factor \\
Target modules & q\_proj, v\_proj & Attention projection layers \\
Learning rate & 3e-4 & Peak learning rate \\
Batch size & 8 & Per-device batch size \\
Gradient accumulation & 4 & Effective batch size: 32 \\
Epochs & 3 & Training epochs per domain \\
Warmup ratio & 0.1 & Learning rate warmup \\
LR scheduler & cosine & Cosine annealing schedule \\
Weight decay & 0.01 & L2 regularization \\
Max sequence length & 2048 & Context window \\
\hline
\end{tabular}
\end{table}

\textbf{Domain-Specific Fine-tuning:}
For each capability domain, we fine-tune the compressed model on the corresponding training datasets and evaluate on held-out test sets. This allows us to measure:
\begin{itemize}
    \item \textbf{Performance Recovery:} How much of the quantization-induced degradation is recovered through fine-tuning
    \item \textbf{Domain Adaptability:} Whether compressed models can still effectively adapt to specialized tasks
    \item \textbf{Training Stability:} If quantization affects gradient flow and optimization dynamics
\end{itemize}

\textbf{Evaluation Protocol:}
\begin{enumerate}
    \item \textbf{Baseline:} Evaluate compressed model before fine-tuning
    \item \textbf{Fine-tune:} Train on domain-specific datasets for 3 epochs
    \item \textbf{Evaluate:} Test on held-out evaluation sets using standard benchmarks
    \item \textbf{Compare:} Measure performance delta vs. FP16 baseline and pre-fine-tuning compressed model
\end{enumerate}

\subsubsection{Rationale for Domain Selection}

The six capability domains are selected to provide comprehensive coverage of model capabilities:

\begin{itemize}
    \item \textbf{Mathematical Reasoning} tests formal symbolic manipulation, which is highly sensitive to quantization due to the precision required in arithmetic operations
    
    \item \textbf{Code Generation} evaluates structured output generation and syntax precision, where small errors can break functionality
    
    \item \textbf{World Knowledge} assesses factual retention in model weights, directly testing whether quantization causes knowledge loss
    
    \item \textbf{Domain Expertise} examines specialized knowledge preservation in professional contexts with technical vocabulary
    
    \item \textbf{Language Understanding} measures fluency and coherence, which can degrade with aggressive compression
    
    \item \textbf{Instruction Following} evaluates controllability and steerability, critical for practical deployment
\end{itemize}

Together, these domains span the spectrum from precise symbolic reasoning (math, code) to open-ended generation (language, summarization), from general knowledge (world facts) to specialized expertise (medical, legal), and from zero-shot evaluation (code, comprehension) to few-shot adaptation (MMLU, GSM8K).

\subsection{RAG Pipeline Configuration}

\subsubsection{Pipeline Architecture}
Our RAG (Retrieval-Augmented Generation) pipeline implements a modular architecture consisting of five core components that work in sequence to enable context-aware question answering:

\begin{enumerate}
    \item \textbf{Document Processing:} Extract and clean text from source documents (PDF, TXT, Markdown)
    \item \textbf{Text Chunking:} Split documents into semantically coherent segments
    \item \textbf{Embedding:} Convert text chunks into dense vector representations
    \item \textbf{Vector Indexing:} Store and index embeddings for efficient similarity search
    \item \textbf{Retrieval \& Generation:} Query the index and generate contextualized answers
\end{enumerate}

The pipeline is implemented as a reusable framework that can be applied to any of our compressed models, enabling direct comparison of RAG performance across quantization methods.

\subsubsection{Document Processing}

\textbf{Text Extraction:}
\begin{itemize}
    \item \textbf{PDF Processing:} PyPDF2-based extraction with page-level granularity
    \item \textbf{Text Normalization:} Whitespace normalization, OCR error correction, quote standardization
    \item \textbf{Cleaning Operations:}
    \begin{itemize}
        \item Remove page numbers and headers
        \item Strip citation references ([1], (Author, 2020))
        \item Remove URLs and excessive whitespace
        \item Fix common ligature errors (fi, fl)
    \end{itemize}
\end{itemize}

\begin{table}[H]
\centering
\caption{Document processing parameters}
\label{tab:doc_processing_params}
\small
\begin{tabular}{lll}
\hline
\textbf{Parameter} & \textbf{Default Value} & \textbf{Description} \\
\hline
remove\_headers & true & Strip page headers and numbers \\
remove\_citations & true & Remove citation markers \\
extract\_sections & false & Parse document sections \\
\hline
\end{tabular}
\end{table}

\subsubsection{Text Chunking Strategy}

We employ a \textbf{semantic chunking} strategy that respects natural document boundaries while maintaining optimal chunk sizes for embedding and retrieval. This approach outperforms fixed-size chunking by preserving contextual coherence.

\textbf{Chunking Algorithm:}
\begin{itemize}
    \item \textbf{Strategy:} Semantic (paragraph-aware)
    \item \textbf{Primary Delimiter:} Double newlines (paragraph boundaries)
    \item \textbf{Fallback:} Sentence-level tokenization using NLTK punkt tokenizer
    \item \textbf{Overlap Mechanism:} Sliding window with configurable overlap to maintain context continuity across chunk boundaries
\end{itemize}

\begin{table}[H]
\centering
\caption{Text chunking parameters}
\label{tab:chunking_params}
\small
\begin{tabular}{llp{6cm}}
\hline
\textbf{Parameter} & \textbf{Default Value} & \textbf{Description} \\
\hline
strategy & semantic & Chunking strategy (semantic, sentence, fixed) \\
chunk\_size & 512 & Target chunk size in tokens \\
chunk\_overlap & 50 & Overlap between consecutive chunks \\
min\_chunk\_size & 100 & Minimum viable chunk size \\
\hline
\end{tabular}
\end{table}

\textbf{Chunk Metadata:}
Each chunk includes metadata for traceability and filtering:
\begin{itemize}
    \item \textbf{chunk\_id:} Unique identifier (chunk\_0, chunk\_1, ...)
    \item \textbf{page\_number:} Source page in original document
    \item \textbf{start\_char / end\_char:} Character offsets in source
    \item \textbf{tokens:} Word count for the chunk
    \item \textbf{section:} Optional section header (if extracted)
\end{itemize}

\subsubsection{Embedding Model}

We use \textbf{sentence-transformers/all-MiniLM-L6-v2} as our embedding model, balancing quality and efficiency for retrieval tasks.

\textbf{Model Characteristics:}
\begin{itemize}
    \item \textbf{Architecture:} Distilled from Microsoft MiniLM
    \item \textbf{Embedding Dimension:} 384
    \item \textbf{Max Sequence Length:} 256 tokens
    \item \textbf{Training:} Fine-tuned on 1B+ sentence pairs
    \item \textbf{Performance:} SBERT benchmark score: 68.06 (semantic similarity)
\end{itemize}

\begin{table}[H]
\centering
\caption{Embedding model parameters}
\label{tab:embedding_params}
\small
\begin{tabular}{llp{6cm}}
\hline
\textbf{Parameter} & \textbf{Default Value} & \textbf{Description} \\
\hline
model\_name & all-MiniLM-L6-v2 & SentenceTransformer model identifier \\
batch\_size & 32 & Batch size for embedding generation \\
normalize & true & L2 normalization of embeddings \\
device & cuda & Compute device (cuda, mps, cpu) \\
\hline
\end{tabular}
\end{table}

\textbf{Implementation:}
\begin{lstlisting}[language=Python, caption={Embedding Generation}]
from sentence_transformers import SentenceTransformer

model = SentenceTransformer(
    'sentence-transformers/all-MiniLM-L6-v2',
    device='cuda'
)

embeddings = model.encode(
    texts,
    batch_size=32,
    show_progress_bar=True,
    normalize_embeddings=True,
    convert_to_numpy=True
)
\end{lstlisting}

\subsubsection{Vector Store and Indexing}

We use \textbf{ChromaDB} as our vector database, providing efficient similarity search with multiple distance metrics.

\textbf{Vector Store Configuration:}
\begin{itemize}
    \item \textbf{Database:} ChromaDB (persistent or in-memory)
    \item \textbf{Index Type:} HNSW (Hierarchical Navigable Small World)
    \item \textbf{Distance Metric:} Cosine similarity (default)
    \item \textbf{Storage:} Optional persistent storage for index reuse
\end{itemize}

\begin{table}[H]
\centering
\caption{Vector store parameters}
\label{tab:vector_store_params}
\small
\begin{tabular}{llp{6cm}}
\hline
\textbf{Parameter} & \textbf{Default Value} & \textbf{Description} \\
\hline
collection\_name & rag\_documents & Index collection identifier \\
persist\_directory & null & Directory for persistent storage (null=in-memory) \\
distance\_metric & cosine & Distance function (cosine, l2, ip) \\
\hline
\end{tabular}
\end{table}

\textbf{Distance-to-Similarity Conversion:}
ChromaDB returns distances that must be converted to similarity scores [0, 1]:
\begin{itemize}
    \item \textbf{Cosine:} $\text{similarity} = 1 - \frac{d^2}{2}$ where $d$ is L2 distance of normalized vectors
    \item \textbf{L2:} $\text{similarity} = \frac{1}{1 + d}$ (exponential decay)
    \item \textbf{Inner Product:} $\text{similarity} = \frac{d + 2}{2}$ (normalized to [0, 1])
\end{itemize}

\subsubsection{Retrieval Strategy}

Our retrieval system implements advanced techniques beyond simple nearest-neighbor search:

\textbf{Base Retrieval:}
\begin{itemize}
    \item \textbf{Top-K Selection:} Retrieve top-3 most similar chunks by default
    \item \textbf{Similarity Threshold:} Configurable minimum similarity score (default: 0.0)
    \item \textbf{Metadata Filtering:} Optional filtering by page number, section, etc.
\end{itemize}

\textbf{Re-ranking (Optional):}
Hybrid retrieval combining semantic and lexical matching:
\begin{itemize}
    \item \textbf{Semantic Score:} Original embedding similarity (70\% weight)
    \item \textbf{Lexical Score:} Token overlap between query and chunk (30\% weight)
    \item \textbf{Formula:} $\text{rerank\_score} = 0.7 \times \text{cosine\_sim} + 0.3 \times \text{token\_overlap}$
\end{itemize}

\textbf{Diversity Mechanism (Optional):}
Maximal Marginal Relevance (MMR) to reduce redundancy:
\begin{itemize}
    \item \textbf{Objective:} Balance relevance and diversity in retrieved chunks
    \item \textbf{Formula:} $\text{MMR} = \lambda \times \text{Sim}(q, c) - (1-\lambda) \times \max[\text{Sim}(c, S)]$
    \item where $q$ is query, $c$ is candidate chunk, $S$ is selected chunks, $\lambda$ is diversity parameter
\end{itemize}

\begin{table}[H]
\centering
\caption{Retrieval parameters}
\label{tab:retrieval_params}
\small
\begin{tabular}{llp{6cm}}
\hline
\textbf{Parameter} & \textbf{Default Value} & \textbf{Description} \\
\hline
top\_k & 3 & Number of chunks to retrieve \\
similarity\_threshold & 0.0 & Minimum similarity score \\
rerank & false & Enable hybrid re-ranking \\
diversity\_penalty & 0.0 & MMR diversity parameter [0, 1] \\
\hline
\end{tabular}
\end{table}

\subsubsection{Answer Generation}

The generation component uses the compressed LLM with retrieved context to produce grounded answers.

\textbf{Prompt Engineering:}
We design prompts to encourage faithful, concise answers based on retrieved context:

\begin{lstlisting}[language=Python, caption={RAG Generation Prompt Template}]
prompt = f"""Use the following context to answer the question. 
Provide a clear, direct answer based on the information given.

Context:
{retrieved_context}

Question: {user_query}

Answer:"""
\end{lstlisting}

\textbf{Generation Parameters:}
Carefully tuned to balance faithfulness and naturalness:
\begin{itemize}
    \item \textbf{Max New Tokens:} 128 (concise answers)
    \item \textbf{Temperature:} 0.3 (low for factual accuracy)
    \item \textbf{Top-p:} 0.9 (nucleus sampling)
    \item \textbf{Repetition Penalty:} 1.15 (prevent loops)
    \item \textbf{Sampling:} Enabled (allows natural phrasing)
\end{itemize}

\begin{table}[H]
\centering
\caption{Answer generation parameters}
\label{tab:generation_params}
\small
\begin{tabular}{llp{6cm}}
\hline
\textbf{Parameter} & \textbf{Default Value} & \textbf{Description} \\
\hline
max\_new\_tokens & 128 & Maximum answer length \\
temperature & 0.3 & Sampling temperature \\
top\_p & 0.9 & Nucleus sampling threshold \\
do\_sample & true & Enable sampling vs greedy \\
repetition\_penalty & 1.15 & Penalty for repeated tokens \\
use\_chat\_template & true & Use model's chat template if available \\
\hline
\end{tabular}
\end{table}

\textbf{Answer Validation:}
Post-processing to ensure quality:
\begin{itemize}
    \item \textbf{Truncation:} Limit to 4 sentences maximum
    \item \textbf{Context Truncation:} Cap context at 2000 characters to prevent overwhelming
    \item \textbf{Retry Mechanism:} If answer appears problematic (too short, repetitive, verbatim copying), retry with simplified prompt and lower temperature (0.2)
    \item \textbf{Fallback:} Return "The information is not provided in the given context" for empty/invalid responses
\end{itemize}

\subsubsection{RAG Evaluation Dataset}

We create a custom technical QA dataset from documentation corpora to evaluate RAG performance:

\textbf{Dataset Characteristics:}
\begin{itemize}
    \item \textbf{Domain:} Technical documentation (ML frameworks, APIs)
    \item \textbf{Size:} 10-50 question-answer pairs per evaluation
    \item \textbf{Question Types:}
    \begin{itemize}
        \item Factual: "What is the default learning rate?"
        \item Procedural: "How do you initialize a model?"
        \item Comparative: "What's the difference between X and Y?"
    \end{itemize}
    \item \textbf{Answer Format:} Short-form answers (1-3 sentences)
    \item \textbf{Ground Truth:} Human-verified reference answers
\end{itemize}

\begin{table}[H]
\centering
\caption{RAG evaluation dataset parameters}
\label{tab:rag_eval_dataset_params}
\small
\begin{tabular}{llp{6cm}}
\hline
\textbf{Parameter} & \textbf{Default Value} & \textbf{Description} \\
\hline
num\_questions & 10 & Number of QA pairs to evaluate \\
dataset\_path & null & Path to custom QA JSON file \\
compare\_no\_rag & true & Evaluate without retrieval baseline \\
save\_detailed\_responses & false & Save individual responses to file \\
\hline
\end{tabular}
\end{table}

\textbf{Evaluation Protocol:}
\begin{enumerate}
    \item Index source documents using the RAG pipeline
    \item For each test question:
    \begin{itemize}
        \item Retrieve top-K relevant chunks
        \item Generate answer with context (RAG)
        \item Generate answer without context (no-RAG baseline)
    \end{itemize}
    \item Compute metrics comparing predictions to reference answers
    \item Aggregate results across all questions
\end{enumerate}

\subsubsection{Pipeline Integration with Compressed Models}

The RAG pipeline is model-agnostic and interfaces with any compressed model through a unified ModelInterface:

\begin{lstlisting}[language=Python, caption={RAG Pipeline Initialization}]
from rag import RAGPipeline

# Load compressed model
model_interface = load_compressed_model("NF4")  

# Initialize RAG pipeline
rag_config = {
    "chunking": {"strategy": "semantic", "chunk_size": 512},
    "embedding": {"model_name": "all-MiniLM-L6-v2"},
    "retrieval": {"top_k": 3, "rerank": False},
    "generation": {"temperature": 0.3, "max_new_tokens": 128}
}

pipeline = RAGPipeline(rag_config)
pipeline.setup(model_interface)

# Index documents
pipeline.index_documents("technical_docs.pdf")

# Evaluate
results = pipeline.evaluate(test_questions, compare_no_rag=True)
\end{lstlisting}

This design enables fair comparison of RAG performance across all compression methods, as all components except the generation model remain constant.

\newpage
\subsection{Evaluation Metrics and Benchmarks}

Our evaluation framework assesses compressed models across three dimensions: computational efficiency, task performance, and retrieval-augmented generation capabilities.

\subsubsection{Efficiency Metrics}
We measure computational efficiency through the following metrics:

\textbf{Latency Measurements:}
\begin{itemize}
    \item \textbf{Average Latency:} Mean time per generated token (ms) measured across multiple inference runs with warmup iterations to ensure stable GPU states
    \item \textbf{Time to First Token (TTFT):} Initial response latency measuring the time from prompt submission to first token generation, critical for interactive applications
    \item \textbf{Prefill vs. Decode Latency:} Separate measurement of prompt processing time (prefill) and autoregressive generation time (decode) to identify optimization bottlenecks
\end{itemize}

\begin{table}[H]
\centering
\caption{Latency measurement parameters}
\label{tab:latency_params}
\small
\begin{tabular}{lll}
\hline
\textbf{Parameter} & \textbf{Default Value} & \textbf{Description} \\
\hline
num\_warmup & 3 & Warmup iterations before measurement \\
num\_runs & 10 & Number of measurement iterations \\
max\_new\_tokens & 128 & Maximum tokens to generate per prompt \\
prompts & 8 prompts & List of test prompts for benchmarking \\
\hline
\end{tabular}
\end{table}

\textbf{Throughput and Memory:}
\begin{itemize}
    \item \textbf{Throughput:} Sustained generation rate measured in tokens per second, averaged over extended generation sequences
    \item \textbf{Peak Memory:} Maximum GPU memory allocated during inference (MB), captured using CUDA memory profiling
    \item \textbf{Model Size:} Disk storage requirements (GB) including all parameters and buffers
    \item \textbf{Memory Efficiency:} Ratio of model size to peak memory usage, indicating memory overhead beyond model parameters (e.g., activations, KV cache)
\end{itemize}

\begin{table}[H]
\centering
\caption{Throughput and memory measurement parameters}
\label{tab:throughput_params}
\small
\begin{tabular}{lll}
\hline
\textbf{Parameter} & \textbf{Default Value} & \textbf{Description} \\
\hline
num\_runs & 10 & Number of measurement iterations \\
max\_new\_tokens & 128 & Tokens to generate per run \\
batch\_size & 1 & Batch size for evaluation \\
measure\_batch\_throughput & false & Test multiple batch sizes \\
batch\_sizes & [1, 2, 4, 8] & Batch sizes to test (if enabled) \\
\hline
\end{tabular}
\end{table}

\textbf{Computational Efficiency:}
\begin{itemize}
    \item \textbf{Model FLOPs Utilization (MFU):} Percentage of theoretical peak hardware FLOPs achieved during inference, calculated as $\text{MFU} = \frac{\text{Achieved FLOPs/s}}{\text{Peak Hardware FLOPs/s}} \times 100\%$, where achieved FLOPs is the product of FLOPs per token and throughput
    \item \textbf{Energy Consumption:} Estimated energy per token (mJ) based on device Thermal Design Power (TDP) and measured latency, using the formula $E = (P_{\text{TDP}} - P_{\text{idle}}) \times t$, where $P_{\text{idle}}$ is assumed to be 30\% of TDP
\end{itemize}

\begin{table}[H]
\centering
\caption{Computational efficiency parameters}
\label{tab:compute_params}
\small
\begin{tabular}{lll}
\hline
\textbf{Parameter} & \textbf{Default Value} & \textbf{Description} \\
\hline
device\_tdp\_watts & Auto-detected & Device thermal design power \\
idle\_power\_ratio & 0.3 & Fraction of TDP at idle (30\%) \\
peak\_tflops & Auto-detected & Hardware peak TFLOPs (FP16) \\
\hline
\end{tabular}
\end{table}

\subsubsection{Performance Benchmarks}
We evaluate model quality using established language modeling benchmarks, organized by capability:

\textbf{Language Modeling:}
\begin{itemize}
    \item \textbf{Perplexity:} Measured on WikiText-2 test set using sliding window evaluation with stride 512 to assess next-token prediction quality. Lower perplexity indicates better language understanding.
\end{itemize}

\begin{table}[H]
\centering
\caption{Perplexity evaluation parameters}
\label{tab:perplexity_params}
\small
\begin{tabular}{lll}
\hline
\textbf{Parameter} & \textbf{Default Value} & \textbf{Description} \\
\hline
dataset & wikitext & HuggingFace dataset name \\
dataset\_config & wikitext-2-raw-v1 & Dataset configuration \\
split & test & Dataset split to evaluate \\
num\_samples & 100 & Number of samples to process \\
max\_length & 512 & Maximum sequence length \\
stride & 512 & Sliding window stride (null=no sliding) \\
batch\_size & 1 & Batch size for processing \\
\hline
\end{tabular}
\end{table}

\textbf{Selected Core Tasks:}

All core tasks use the Language Model Evaluation Harness \cite{eval-harness} with consistent hyperparameters for reproducibility. We report accuracy (or pass@1 for code tasks) normalized to [0,1].

\begin{table}[H]
\centering
\caption{Core task benchmark parameters}
\label{tab:core_tasks_params}
\small
\begin{tabular}{lllp{4.5cm}}
\hline
\textbf{Task} & \textbf{Few-Shot} & \textbf{Metric} & \textbf{Description} \\
\hline
HellaSwag & 0 & acc\_norm & Commonsense reasoning via sentence completion in everyday scenarios \\
\hline
ARC-Easy & 0 & acc\_norm & Grade-school level science question answering \\
\hline
ARC-Challenge & 0 & acc\_norm & Challenge-level scientific reasoning questions \\
\hline
GSM8K & 8 & exact\_match & Grade-school math word problems with step-by-step reasoning \\
\hline
MMLU & 5 & acc & Multi-domain knowledge across 57 academic subjects \\
\hline
HumanEval & 0 & pass@1 & Python code generation evaluated on test case pass rate \\
\hline
\end{tabular}
\end{table}

\begin{table}[H]
\centering
\caption{LM-Eval harness global parameters}
\label{tab:lm_eval_global_params}
\small
\begin{tabular}{lll}
\hline
\textbf{Parameter} & \textbf{Default Value} & \textbf{Description} \\
\hline
batch\_size & 1 & Global batch size for all tasks \\
limit & null & Limit samples per task (null=all) \\
random\_seed & 1234 & Random seed for reproducibility \\
\hline
\end{tabular}
\end{table}

\textbf{Original Mistral-7B Benchmarks:}

\begin{table}[H]
\centering
\caption{Mistral-7B complete benchmark suite}
\label{tab:mistral_full_suite}
\footnotesize
\begin{tabular}{llll}
\hline
\textbf{Category} & \textbf{Tasks} & \textbf{Few-Shot} & \textbf{Metric} \\
\hline
\multirow{8}{*}{\begin{tabular}[c]{@{}l@{}}Commonsense\\Reasoning\end{tabular}} 
& HellaSwag & 0 & acc\_norm \\
& Winogrande & 0 & acc \\
& PIQA & 0 & acc\_norm \\
& SIQA & 0 & acc \\
& OpenbookQA & 0 & acc\_norm \\
& ARC-Easy & 0 & acc\_norm \\
& ARC-Challenge & 0 & acc\_norm \\
& CommonsenseQA & 0 & acc \\
\hline
\multirow{2}{*}{\begin{tabular}[c]{@{}l@{}}World\\Knowledge\end{tabular}}
& NaturalQuestions & 5 & exact\_match \\
& TriviaQA & 5 & exact\_match \\
\hline
\multirow{2}{*}{\begin{tabular}[c]{@{}l@{}}Reading\\Comprehension\end{tabular}}
& BoolQ & 0 & acc \\
& QuAC & 0 & f1 \\
\hline
\multirow{2}{*}{Math}
& GSM8K & 8 (maj@8) & exact\_match \\
& MATH & 4 (maj@4) & exact\_match \\
\hline
\multirow{2}{*}{Code}
& HumanEval & 0 & pass@1 \\
& MBPP & 3 & pass@1 \\
\hline
\multirow{3}{*}{\begin{tabular}[c]{@{}l@{}}Aggregate\\Benchmarks\end{tabular}}
& MMLU & 5 (57 tasks) & acc \\
& BBH & 3 (23 tasks) & acc \\
& AGI Eval & 3-5 & acc \\
\hline
\end{tabular}
\end{table}

\subsubsection{RAG Evaluation}

For retrieval-augmented generation, we evaluate both retrieval quality and answer generation using a custom question-answering dataset.

\textbf{Retrieval Quality Metrics:}

\begin{table}[H]
\centering
\caption{Retrieval quality metrics and parameters}
\label{tab:retrieval_metrics_params}
\small
\begin{tabular}{lp{8cm}}
\hline
\textbf{Metric} & \textbf{Description} \\
\hline
Context Sufficiency & Fraction of queries where retrieved contexts contain sufficient information (80\% token overlap threshold) \\
Context Precision & Relevance of retrieved chunks measured by query-context token overlap \\
Context Coverage & Fraction of answer tokens present in retrieved contexts \\
Retrieval Consistency & Standard deviation of retrieval scores, indicating stability \\
Precision@K & Fraction of top-K retrieved items that are relevant \\
Recall@K & Fraction of relevant items in top-K retrieved \\
F1@K & Harmonic mean of Precision@K and Recall@K \\
MRR & Mean reciprocal rank of first relevant item \\
MAP & Mean average precision across all queries \\
\hline
\end{tabular}
\end{table}

\begin{table}[H]
\centering
\caption{Retrieval evaluation parameters}
\label{tab:retrieval_eval_params}
\small
\begin{tabular}{lll}
\hline
\textbf{Parameter} & \textbf{Default Value} & \textbf{Description} \\
\hline
top\_k & 3 & Number of chunks to retrieve \\
k\_values & [1, 3, 5, 10] & K values for precision@k, recall@k \\
similarity\_threshold & 0.3 & Minimum similarity score threshold \\
relevance\_token\_threshold & 0.3 & Token overlap threshold for relevance \\
sufficiency\_token\_threshold & 0.8 & Token overlap threshold for sufficiency \\
\hline
\end{tabular}
\end{table}

\textbf{Answer Generation Metrics:}

\begin{table}[H]
\centering
\caption{Answer generation metrics and parameters}
\label{tab:answer_metrics_params}
\small
\begin{tabular}{lp{8cm}}
\hline
\textbf{Metric} & \textbf{Description} \\
\hline
Exact Match (EM) & Binary correctness: perfect normalized string match \\
F1 Score & Token-level precision-recall harmonic mean \\
Answer Relevance & Query-answer token overlap (measures if answer addresses question) \\
Faithfulness & Token containment: fraction of answer tokens in retrieved context \\
ROUGE-1/2/L & N-gram overlap (unigram, bigram) and longest common subsequence \\
BERTScore & Semantic similarity using contextual BERT embeddings (F1) \\
BLEU & Translation-style matching with smoothing \\
\hline
\end{tabular}
\end{table}

\begin{table}[H]
\centering
\caption{Answer generation parameters}
\label{tab:answer_gen_params}
\small
\begin{tabular}{lll}
\hline
\textbf{Parameter} & \textbf{Default Value} & \textbf{Description} \\
\hline
max\_new\_tokens & 128 & Maximum tokens in generated answer \\
temperature & 0.3 & Sampling temperature (lower=deterministic) \\
top\_p & 0.9 & Nucleus sampling threshold \\
repetition\_penalty & 1.15 & Penalty for repeated tokens \\
normalize\_whitespace & true & Normalize whitespace in comparisons \\
case\_sensitive & false & Case-sensitive matching \\
remove\_punctuation & false & Remove punctuation before comparison \\
rouge\_use\_stemmer & true & Use Porter stemmer for ROUGE \\
bertscore\_lang & en & Language for BERTScore \\
\hline
\end{tabular}
\end{table}

\textbf{RAG Efficiency:}

\begin{table}[H]
\centering
\caption{RAG efficiency metrics}
\label{tab:rag_efficiency_metrics}
\small
\begin{tabular}{lp{9cm}}
\hline
\textbf{Metric} & \textbf{Description} \\
\hline
Retrieval Time & Average time to retrieve top-k contexts (ms) \\
RAG Generation Time & Time to generate answers with retrieved context (ms) \\
No-RAG Generation Time & Baseline generation time without context (ms) \\
RAG Throughput & Generation speed with context (tokens/sec) \\
No-RAG Throughput & Generation speed without context (tokens/sec) \\
Generation Speedup & Ratio of no-RAG to RAG generation time \\
F1 Improvement & Delta between RAG and no-RAG F1 scores \\
EM Improvement & Delta between RAG and no-RAG exact match scores \\
\hline
\end{tabular}
\end{table}

\begin{table}[H]
\centering
\caption{RAG evaluation dataset parameters}
\label{tab:rag_dataset_params}
\small
\begin{tabular}{lll}
\hline
\textbf{Parameter} & \textbf{Default Value} & \textbf{Description} \\
\hline
num\_questions & 10 & Number of QA pairs to evaluate \\
dataset\_path & null & Path to custom QA dataset (JSON) \\
compare\_no\_rag & true & Compare with no-RAG baseline \\
save\_detailed\_responses & false & Save individual Q\&A responses \\
\hline
\end{tabular}
\end{table}

All RAG metrics are averaged over the evaluation dataset sampled from a technical documentation corpus, with retrieval configured to return top-3 chunks per query by default.


\newpage
\section{Results}

\subsection{Overview}
We present a comprehensive comparison of compression techniques across three dimensions: efficiency gains, performance preservation, and RAG capabilities. Our analysis focuses on understanding the trade-offs between model size reduction, inference speed, and task performance for each compression method.

\subsection{Compression Efficiency Analysis}

\subsubsection{Quantization Methods}

\begin{table}[H]
\centering
\caption{Performance comparison of quantization methods across benchmarks}
\label{tab:quant_performance}
\small
\begin{tabular}{lccccccc}
\hline
\textbf{Method} & \textbf{Perplexity} & \textbf{HellaSwag} & \textbf{ARC-Easy} & \textbf{ARC-Challenge} & \textbf{GSM8K} & \textbf{MMLU} & \textbf{HumanEval} \\
 & \textit{(↓)} & \textit{(0-shot)} & \textit{(0-shot)} & \textit{(0-shot)} & \textit{(8-shot)} & \textit{(5-shot)} & \textit{(0-shot)} \\
\hline
FP16 & 12.79 & 0.72 & 0.76 & 0.58 & 0.36 & 1.00 & 0.05 \\
NF4 & 13.02 & 0.70 & 0.75 & 0.58 & 0.27 & 0.55 & 0.05 \\
GPTQ & 12.85 & 0.68 & 0.75 & 0.60 & --- & --- & 0.05 \\
AWQ & 13.47 & --- & --- & --- & --- & --- & --- \\
HQQ & 13.5 & 0.69 & 0.72 & 0.5 & --- & --- & --- \\
\hline
\end{tabular}
\end{table}

\begin{table}[H]
\centering
\caption{Average accuracy and performance degradation for quantization methods}
\label{tab:quant_performance_summary}
\small
\begin{tabular}{lcccc}
\hline
\textbf{Method} & \textbf{Average} & \textbf{Perplexity} & \textbf{Accuracy} & \textbf{Tasks} \\
 & \textbf{Accuracy} & \textbf{Increase} & \textbf{Drop} & \textbf{Evaluated} \\
\hline
FP16 & 0.57 & --- & --- & 6 \\
NF4 & 0.52 & +1.80\% & -0.05 & 6 \\
GPTQ & 0.52 & +0.47\% & -0.05 & 6 \\
AWQ & --- & +5.32\% & --- & 6 \\
HQQ & --- & --- & --- & 6 \\
\hline
\end{tabular}
\end{table}

\newpage
\subsection{Performance Preservation Analysis}

\subsubsection{Quantization Methods}

\begin{table}[H]
\centering
\caption{RAG answer quality evaluation of quantization methods}
\label{tab:quant_rag}
\small
\begin{tabular}{lcccccccc}
\hline
\textbf{Method} & \textbf{F1} & \textbf{EM} & \textbf{Faithful-} & \textbf{Relevance} & \textbf{ROUGE-1} & \textbf{ROUGE-2} & \textbf{ROUGE-L} & \textbf{BERTScore} \\
 & & & \textbf{ness} & & & & & \textbf{F1} \\
\hline
FP16 & 0.217 & 0.0 & 0.559 & 0.094 & 0.279 & 0.092 & 0.197 & 0.658 \\
NF4 & 0.205 & 0.0 & 0.539 & 0.083 & 0.244 & 0.086 & 0.177 & 0.614 \\
GPTQ & --- & --- & --- & --- & --- & --- & --- & --- \\
AWQ & 0.191 & 0.0 & 0.476 & 0.082 & 0.243 & 0.088 & 0.181 & 0.615 \\
HQQ & --- & --- & --- & --- & --- & --- & --- & --- \\
\hline
\end{tabular}
\end{table}

\begin{table}[H]
\centering
\caption{RAG vs no-RAG comparison for quantization methods}
\label{tab:quant_rag_comparison}
\small
\begin{tabular}{lcccccc}
\hline
\textbf{Method} & \textbf{No-RAG} & \textbf{No-RAG} & \textbf{F1} & \textbf{EM} & \textbf{Avg Answer} & \textbf{Avg Answer} \\
 & \textbf{F1} & \textbf{EM} & \textbf{Improvement} & \textbf{Improvement} & \textbf{Length (RAG)} & \textbf{Length (No-RAG)} \\
\hline
FP16 & 0.190 & 0.0 & +0.027 & 0.0 & 33.75 & --- \\
NF4 & 0.181 & 0.0 & +0.024 & 0.0 & 31.95 & --- \\
GPTQ & --- & --- & --- & --- & --- & --- \\
AWQ & 0.165 & 0.0 & +0.025 & 0.0 & 30.5 & --- \\
HQQ & --- & --- & --- & --- & --- & --- \\
\hline
\end{tabular}
\end{table}

\newpage
\subsection{RAG Performance Analysis}

\subsubsection{Quantization Methods}

\begin{table}[H]
\centering
\caption{RAG answer quality evaluation of quantization methods}
\label{tab:quant_rag}
\small
\begin{tabular}{lcccccccc}
\hline
\textbf{Method} & \textbf{F1} & \textbf{EM} & \textbf{Faithful-} & \textbf{Relevance} & \textbf{ROUGE-1} & \textbf{ROUGE-2} & \textbf{ROUGE-L} & \textbf{BERTScore} \\
 & & & \textbf{ness} & & & & & \textbf{F1} \\
\hline
FP16 & 0.217 & 0.0 & 0.559 & 0.094 & 0.279 & 0.092 & 0.197 & 0.658 \\
NF4 & 0.205 & 0.0 & 0.539 & 0.083 & 0.244 & 0.086 & 0.177 & 0.614 \\
GPTQ & --- & --- & --- & --- & --- & --- & --- & --- \\
AWQ & 0.191 & 0.0 & 0.476 & 0.082 & 0.243 & 0.088 & 0.181 & 0.615 \\
HQQ & --- & --- & --- & --- & --- & --- & --- & --- \\
\hline
\end{tabular}
\end{table}

\begin{table}[H]
\centering
\caption{RAG vs no-RAG comparison for quantization methods}
\label{tab:quant_rag_comparison}
\small
\begin{tabular}{lcccccc}
\hline
\textbf{Method} & \textbf{No-RAG} & \textbf{No-RAG} & \textbf{F1} & \textbf{EM} & \textbf{Avg Answer} & \textbf{Avg Answer} \\
 & \textbf{F1} & \textbf{EM} & \textbf{Improvement} & \textbf{Improvement} & \textbf{Length (RAG)} & \textbf{Length (No-RAG)} \\
\hline
FP16 & 0.190 & 0.0 & +0.027 & 0.0 & 33.75 & --- \\
NF4 & 0.181 & 0.0 & +0.024 & 0.0 & 31.95 & --- \\
GPTQ & --- & --- & --- & --- & --- & --- \\
AWQ & 0.165 & 0.0 & +0.025 & 0.0 & 30.5 & --- \\
HQQ & --- & --- & --- & --- & --- & --- \\
\hline
\end{tabular}
\end{table}

\begin{table}[H]
\centering
\caption{Context retrieval quality across quantization methods}
\label{tab:quant_retrieval}
\small
\begin{tabular}{lccccccc}
\hline
\textbf{Method} & \textbf{Sufficiency} & \textbf{Precision} & \textbf{Coverage} & \textbf{Consistency} & \textbf{Avg Score} & \textbf{Avg Chunks} & \textbf{Avg Context} \\
 & & & & \textbf{(std)} & & \textbf{Retrieved} & \textbf{Length} \\
\hline
FP16 & 0.796 & 0.564 & 0.756 & 0.090 & 0.800 & 3.0 & 1308.5 \\
NF4 & 0.796 & 0.564 & 0.756 & 0.090 & 0.800 & 3.0 & 1308.5 \\
GPTQ & --- & --- & --- & --- & --- & --- & --- \\
AWQ & 0.796 & 0.564 & 0.756 & 0.090 & 0.800 & 3.0 & 1308.5 \\
HQQ & --- & --- & --- & --- & --- & --- & --- \\
\hline
\end{tabular}
\end{table}

\begin{table}[H]
\centering
\caption{RAG efficiency metrics for quantization methods}
\label{tab:quant_rag_efficiency}
\small
\begin{tabular}{lcccccc}
\hline
\textbf{Method} & \textbf{Retrieval} & \textbf{RAG Gen} & \textbf{No-RAG Gen} & \textbf{RAG} & \textbf{No-RAG} & \textbf{Generation} \\
 & \textbf{Time (ms)} & \textbf{Time (ms)} & \textbf{Time (ms)} & \textbf{Throughput} & \textbf{Throughput} & \textbf{Speedup} \\
 & & & & \textbf{(tok/s)} & \textbf{(tok/s)} & \\
\hline
FP16 & 24.7 & 8435.5 & 7855.2 & 5.33 & 9.34 & 0.93x \\
NF4 & 28.0 & 10443.9 & 9965.0 & 4.37 & 7.06 & 0.95x \\
GPTQ & --- & --- & --- & --- & --- & --- \\
AWQ & 26.1 & 9993.9 & 7744.2 & 4.36 & 7.94 & 0.77x \\
HQQ & --- & --- & --- & --- & --- & --- \\
\hline
\end{tabular}
\end{table}

\end{document}

\end{document}
